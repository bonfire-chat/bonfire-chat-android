\documentclass[accentcolor=tud2d,colorbacktitle]{tudbeamer}

\usepackage{ngerman}
\usepackage[utf8]{inputenc}
\usepackage{graphicx}
\usepackage{multicol}
\usepackage[most]{tcolorbox}

\title{Entwicklung eines adaptiven Chat-Systems\\}
\author{Johannes Lauinger, Jonas Mönnig, Matthias Hofmann, Max Weller, Simon-Konstantin Thiem}
\date{17.07.2015}

\newtcolorbox{mybox}[2][]{
  fit height from=2.8cm to 2.8cm,
  arc=0pt, outer arc=0pt,
  enhanced,
  colback=#2!40,
  colframe=black!40
}

\begin{document}

\begin{titleframe}
    \textbf{Team:}\\
    Johannes Lauinger, Jonas Mönnig, Matthias Hofmann, Max Weller, Simon-Konstantin Thiem\\
    \vspace{0.5cm}
    \textbf{Teamleiter:}\\
    Benjamin Tumele
    \begin{flushright}
      \includegraphics[height=3.5cm]{icon.png}
    \end{flushright}
  \end{titleframe}

  \begin{frame}
    \frametitle{Was ist BonfireChat?}
    \begin{itemize}
      \item Hybride Chat-Anwendung: Peer-to-peer und Internet
      \item Bluetooth und WiFi Direct
      \item Routing durch AdHoc-Mesh-Netzwerk
      \item Kontaktaustausch dezentral: NFC, QR-Codes, Internet als Fallback
      \item Android App
    \end{itemize}
    \begin{flushright}
      \includegraphics[height=2.5cm]{sgf.jpg}
    \end{flushright}
  \end{frame}

  \begin{frame}
    \frametitle{Projektfortschritt}
    \begin{multicols}{2}
    \begin{mybox}{red}
        \begin{itemize}
          \item Nachrichten verschicken
            \includegraphics[height=1cm]{bluetooth.png}
            \includegraphics[height=1cm]{wifi.png}
            \includegraphics[height=1cm]{cloud.png}
          \item Flooding der Nachrichten
        \end{itemize}
      \end{mybox}
      \begin{mybox}{blue}
        \begin{itemize}
          \item Kontaktdaten austauschen
            \includegraphics[height=1cm]{nfc.png}
            \includegraphics[height=1cm]{qrcode.png}
        \end{itemize}
      \end{mybox}
      \begin{mybox}{green}
        \begin{itemize}
          \item Ende-zu-Ende-Verschlüsselung
            \includegraphics[height=1cm]{lock.png}
        \end{itemize}
      \end{mybox}
      \begin{mybox}{yellow}
        \begin{itemize}
          \item GUI
          \item Dashboard mit Statistiken
        \end{itemize}
      \end{mybox}
    \end{multicols}
  \end{frame}

  \begin{frame}
    \frametitle{Umsetzbarkeit}
    \begin{itemize}
      \item intelligentes Routing
      \item Energieeffizienz
      \item unterstützte Technologie der Handys
    \end{itemize}
    \begin{itemize}
      \item insgesamt aber voraussichtlich umsetzbar
    \end{itemize}
  \end{frame}

  \begin{frame}
    \frametitle{User Interface}
    \begin{center}
      \includegraphics[height=5.5cm]{ui1.jpg}
    \end{center}
  \end{frame}

  \begin{frame}
    \frametitle{User Interface}
    \begin{center}
      \includegraphics[height=6cm]{ui2.jpg}
    \end{center}
  \end{frame}

  \begin{frame}
    \frametitle{Dashboard mit Statistiken}
    \begin{center}
      \includegraphics[height=6.5cm]{dashboard.png}
    \end{center}
  \end{frame}

  \begin{frame}
    \frametitle{Qualitätsziele}
    \begin{itemize}
      \item Wartbarkeit durch Codequalität
      \item Korrektheit
      \item Robustheit
    \end{itemize}
  \end{frame}

  \begin{frame}
    \frametitle{Wartbarkeit durch Codequalität}
    \begin{itemize}
      \item Auftraggeber werden die App weiterentwickeln
      \item Einsatz beim Schlossgrabenfest 2016
    \end{itemize}
    \begin{itemize}
      \item Maßnahmen:
        \begin{itemize}
          \item Checkstyle: automatische Berichte
          \item Code Reviews
          \item technische Dokumentation
        \end{itemize}
    \end{itemize}
  \end{frame}

  \begin{frame}
    \frametitle{Korrektheit}
    \begin{itemize}
      \item Benutzer erwarten korrekte Zustellung ihrer Nachrichten
      \item Ende-zu-Ende-Verschlüsselung
      \item lokale Speicherung
    \end{itemize}
    \begin{itemize}
      \item Maßnahmen:
        \begin{itemize}
          \item automatisierte JUnit Tests
          \item FindBugs
        \end{itemize}
    \end{itemize}
  \end{frame}

  \begin{frame}
    \frametitle{Robustheit}
    \begin{itemize}
      \item für außergewöhnliche Situationen entwickelt
      \item hohe Fehlertoleranz
      \item Sonderfälle beachten
    \end{itemize}
    \begin{itemize}
      \item Maßnahmen: regelmäßige manuelle Tests
    \end{itemize}
  \end{frame}

  \begin{frame}
    \frametitle{Aufgewendete Arbeitszeit}
    \begin{center}
      \includegraphics[height=4.5cm]{zeiten.jpg}
    \end{center}
  \end{frame}

  \begin{frame}
    \frametitle{Aufgewendete Arbeitszeit}
    \begin{center}
      \huge{Insgesamt: 425 Stunden}
      \includegraphics[height=5cm]{zeiten2.jpg}
    \end{center}
  \end{frame}


  \begin{frame}
    \frametitle{Bachelorpraktikum BonfireChat}
    \begin{center}
      \huge{Fragen \& Antworten}
    \end{center}
    \vspace{0.5cm}
    \begin{multicols}{2}
      \begin{itemize}
        \item Was ist BonfireChat?
          \begin{itemize}
            \item Peer-to-peer und Internet
            \item Bluetooth und WiFi Direct
            \item Routing durch AdHoc-Mesh-Netzwerk
            \item Kontaktaustausch: NFC, QR-Codes, Internet als Fallback
          \end{itemize}
        \item Qualitätsziele
          \begin{itemize}
            \item Wartbarkeit durch Codequalität
            \item Korrektheit
            \item Robustheit
          \end{itemize}
          \vspace{1cm}
      \end{itemize}
    \end{multicols}
  \end{frame}

\end{document}
