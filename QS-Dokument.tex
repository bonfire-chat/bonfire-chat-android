\documentclass[accentcolor=tud0b,12pt,paper=a4]{tudreport}

\usepackage[utf8]{inputenc}
\usepackage{ngerman}
\usepackage{parcolumns}

\newcommand{\titlerow}[2]{
	\begin{parcolumns}[colwidths={1=.15\linewidth}]{2}
		\colchunk[1]{#1:} 
		\colchunk[2]{#2}
	\end{parcolumns}
	\vspace{0.2cm}
}

\title{Projektthema}
\subtitle{Qualitätssicherungsdokument}
\subsubtitle{%
	\titlerow{Gruppe 9}{%
		Johannes Lauinger <johannes.lauinger@gmail.com>\\
		Jonas Mönnig <jonas.moennig@gmx.de>\\
		Mattias Hofmann <chaterMH@web.de>\\
		Maximilian Weller <maximilian.weller@stud.tu-darmstadt.de>\\
                Simon-Konstantin Thiem <simon-konstantin.thiem@stud.tu-darmstadt.de>}
	\titlerow{Teamleiter}{Benjamin Tumele <B.Tumele@gmx.de>}
	\titlerow{Auftraggeber}{%
		Alexander Frömmgen <froemmge@dvs.tu-darmstadt.de>\\
		Fachgebiet Datenbanken und Verteilte Systeme\\
		Fachbereich Informatik\\
                Jens Heuschkel <jens.heuschkel@tk.informatik.tu-darmstadt.de>\\
                Fachgebiet Telekooperation\\
                Fachbereich Informatik}
	\titlerow{Abgabedatum}{xx.xx.xxxx}}
\institution{Bachelor-Praktikum SoSe~2015\\Fachbereich Informatik}

\begin{document}

	\maketitle
	\tableofcontents 
	
	\chapter{Einleitung}
        Ziel des Projektes ist die Entwicklung eines adaptiven Chatsystems für Android, mit dem man sowohl über Bluetooth oder WiFi peer-to-peer als auch über eiene Client-Server Infrastruktur kommunizieren kann. Die App soll 2016 auf dem Schlossgrabenfest benutzt werden, um Daten über die Ausfallsicherheit von Netzen zu sammeln. Hintergrund ist, dass auf Großereignissen wie dem Schlossgrabenfest die Handynetze extrem überlastet sind, sodass fast kein Besucher Zugang zum Internet und Telefonnetz hat. Dies macht die entfernte Kommmunikation umöglich.\\

        Wir wollen nun eine Plattform schaffen, über die man peer-to-peer, also unabhängig von zentralen Netzen, kommunizieren kann. Sollte keine Internetverbindung möglich sein, sollen Netzwerkpakete über das peer-to-peer Netzwerk weitergereicht werden, bis ein Client eine Internetverbindung hat und das Paket an den Server übermitteln kann. So ist auch Kommunikation mit Chatpartnern, die sich nicht im eigenen peer-to-peer Netzwerk befinden, möglich. Die App soll vom Funktionsumfang gängigen Messengerapps ähneln, vorerst beschränken wir uns aber bei der Entwicklung auf einfache Textnachrichten in normalen eins-zu-eins-Chats.
	
	\chapter{Qualitätsziele}
        \section{Wartbarkeit}
    
		Da unsere Auftraggeber die fertige App nächstes Jahr auf dem Schlossgrabenfest einsetzen wollen um Daten über die Ausfallsicherheit von Netzen zu sammeln und sie dafür möglicherweise weiterentwickeln ist es notwendig, dass unser Programm wartbar ist.\\

		Zur Sicherstellung der Wartbarkeit werden wir Checkstyle verwenden, um einen einheitlichen Codestil zu gewährleisten.\\
                Checkstyle wird bei jedem Build automatisch ausgeführt werden. Sollte Checkstyle eine Stilverletzung melden, wird der Entwickler, bei dem die Fehlermeldung auftrat, den Codestil anpassen, sodass dieser der Vorgabe entspricht.\\

                Weiterhin soll einmal pro Sprint ein Code-Review stattfinden. Dafür wird das Tool Gerrit verwendet, das kollaborative Code-Reviews unterstützt und sich in Git integriert.\\

                Findbugs verwenden\ldots etc. Es könnte noch jemand anderes was hierzu schreiben :-)

        \section{Datensicherheit}

        \section{Robustheit}

        \section{Energieeffizienz}
	        
	
\appendix	
	\chapter{Anhang}
		(Am Ende des Projekts nachzureichen)\\
		Beleg für durchgeführte Maßnahmen, bzw. falls nicht durchgeführt eine Begründung wieso die Durchführung nicht möglich oder nicht erfolgt ist. \\
		Weitere Anforderungen sind den Unterlagen und der Vorlesung zur Projektbegleitung zu entnehmen.
	
\end{document}
