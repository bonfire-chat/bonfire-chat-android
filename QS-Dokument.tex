\documentclass[accentcolor=tud0b,12pt,paper=a4]{tudreport}

\usepackage[utf8]{inputenc}
\usepackage{ngerman}
\usepackage{parcolumns}

% clickable hyperlinks in the table of contents
\usepackage{hyperref}

\newcommand{\titlerow}[2]{
  \begin{parcolumns}[colwidths={1=.15\linewidth}]{2}
    \colchunk[1]{#1:}
    \colchunk[2]{#2}
  \end{parcolumns}
  \vspace{0.2cm}
}

\title{Entwicklung eines adaptiven Chat-Systems}
\subtitle{Qualitätssicherungsdokument}
\subsubtitle{%
  \titlerow{Gruppe 9}{%
    Johannes Lauinger <johannes.lauinger@gmail.com>\\
    Jonas Mönnig <jonas.moennig@gmx.de>\\
    Mattias Hofmann <chaterMH@web.de>\\
    Maximilian Weller <maximilian.weller@stud.tu-darmstadt.de>\\
    Simon-Konstantin Thiem <simon-konstantin.thiem@stud.tu-darmstadt.de>}
  \titlerow{Teamleiter}{Benjamin Tumele <B.Tumele@gmx.de>}
  \titlerow{Auftraggeber}{%
    Alexander Frömmgen <froemmge@dvs.tu-darmstadt.de>\\
    Fachgebiet Datenbanken und Verteilte Systeme\\
    Fachbereich Informatik\\
    Jens Heuschkel <jens.heuschkel@tk.informatik.tu-darmstadt.de>\\
    Fachgebiet Telekooperation\\
    Fachbereich Informatik}
  \titlerow{Abgabedatum}{xx.xx.xxxx}}
\institution{Bachelor-Praktikum SoSe~2015\\Fachbereich Informatik}

\begin{document}

  \maketitle
  \tableofcontents

  \chapter{Einleitung}
    Mobile Kommunikation ist in der heutigen Zeit selbstverständlich geworden. Wir verlassen uns mehr und mehr darauf, auch von unterwegs jederzeit Nachrichten senden und empfangen zu können. Hierfür wird zum größten Teil das Handynetz verwendet.\\

    Problematisch ist dabei, dass das Mobilfunknetz auf einen bestimmten Frequenzbereich beschränkt ist und daher nicht beliebig viele Nachrichten parallel verschickt werden können. Insbesondere bei Veranstaltungen, bei denen sich sehr viele Menschen auf relativ geringem Platz versammeln, kommt es häufig zu Netzausfällen. Ein Beispiel ist das Darmstädter Schlossgrabenfest, auf dem in den letzten Jahren ca. 100 000 Menschen anwesend waren.\\

    Ziel unseres Projektes ist es, eine Nachrichtenplattform für solche Extremsituationen zu schaffen. Diese soll zusätzlich zur aktuell bereits häufig eingesetzten Möglichkeit, über eine Client-Server-Infrastruktur zu kommunizieren, ein Peer-to-Peer Netzwerk einsetzen. Dies bedeutet, dass die Geräte untereinander ein Netzwerk aufbauen, welches unabhängig von zentralen Netzen funktioniert, und über das Nachrichten verschickt werden können. Kann eine Nachricht nicht unmittelbar zum Empfänger geschickt werden, können weitere Geräte die Nachricht weiterleiten, bis sie am Ziel ankommt.\\

    Die App soll sich in ihrem Funktionsumfang an verbreiteten Messenger-Anwendungen wie zum Beispiel WhatsApp orientieren. Vorerst beschränken wir uns jedoch auf einfache Textnachrichten in Chats mit nur einem Kommunikationspartner. Außerdem wird die App zunächst nur für Android entwickelt.

  \chapter{Qualitätsziele}
    In Zusammenarbeit mit unseren Auftraggebern haben wir beschlossen, bei der Entwicklung der Anwendung besonders auf Wartbarkeit, Korrektheit und Robustheit zu achten. Im Folgenden sollen nun diese Qualitätsziele und die Maßnahmen zur Sicherstellung genauer vorgestellt werden.

    \section{Wartbarkeit}

    Da die Auftraggeber die fertige App nächstes Jahr auf dem Schlossgrabenfest einsetzen wollen, um Daten über die Ausfallsicherheit von Netzen zu sammeln und sie dafür möglicherweise weiterentwickeln, ist es notwendig, dass unser Programm wartbar ist. Darüber hinaus erhöht eine gute Wartbarkeit die Produktivität durch kürzere Wartungszeiten, wodurch der Projektfortschritt beschleunigt wird.\\

                Zur Sicherstellung der Wartbarkeit ist es unter anderem wichtig, auf Lesbarkeit und geringe Codekomplexität zu achten. Dabei unterstützt uns das statische Codeanalysetool Checkstyle. Checkstyle überprüft den Code auf die Einhaltung verschiedenster Programmierrichtlinien und generiert automatisch Berichte. Besonderen Fokus beim Auswerten der Berichte wollen wir dabei auf die von Checkstyle errechneten Codemetriken legen, da diese in direktem Zusammenhang zur Codekomplexität, und daher zur Wartbarkeit stehen. Die Javadoc-Überprüfungen von Checkstyle werden wir jedoch nicht durchführen lassen, da die Auftraggeber keinen Wert auf ausführliches Javadoc legen.\\

                Die Ausführung von Checkstyle automatisieren wir mit der Integration in das Continuous-Integration-Tool Jenkins. Somit werden die Berichte bei jedem Push in das Git-Repostitory erstellt, und sind für jeden einsehbar. Sollten Probleme auftreten, wird im Team entschieden, wer damit beauftragt wird, diese zu beheben.\\

                Weiterhin werden wir das statische Codeanalysetool Findbugs einsetzen, um Fehler im Code frühzeitig zu entdecken. Findbugs integrieren wir ebenfalls in Jenkins, sodass bei jedem Push ins Repository ein Bericht erstellt wird. Werden Fehler gefunden, wird im Team entschieden, wer diese behebt. Da auf diese Art und Weise sofort Fehler automatisch gefunden werden, muss man später keine Zeit mehr dafür aufwenden, diese zu suchen. Diese Maßnahme reduziert somit den Wartungsaufwand und verbessert folglich die Wartbarkeit.\\

                Außerdem wollen die Auftraggeber mit uns zusammen einige Code-Reviews durchführen. Die Modalitäten dieser Reviews sind noch nicht bekannt, aber vermutlich werden die Auftraggeber sich für einen Over-the-Shoulder-Ansatz entscheiden, bei dem sie sich den Code von einem Entwickler erläutern lassen und Anmerkungen und Fragen einbringen. So werden einerseits Schwächen aufgedeckt und behoben, was unter anderem die Wartbarkeit unterstützt, als auch die Codequalität im Allgemeinen gesteigert und die Zufriedenheit der Auftraggeber erhöht, da diese direkten Einfluss auf den Code haben.

                \newpage
    \section{Datensicherheit}

    \section{Robustheit}

\appendix
  \chapter{Anhang}
    (Am Ende des Projekts nachzureichen)\\
    Beleg für durchgeführte Maßnahmen, bzw. falls nicht durchgeführt eine Begründung wieso die Durchführung nicht möglich oder nicht erfolgt ist. \\
    Weitere Anforderungen sind den Unterlagen und der Vorlesung zur Projektbegleitung zu entnehmen.

\end{document}
