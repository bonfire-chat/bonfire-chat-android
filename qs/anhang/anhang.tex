
    Dieser Anhang wurde am Ende des Projekts nachgereicht. Er enthält Belege für
    durchgeführte Maßnahmen, bzw. falls nicht durchgeführt eine Begründung wieso
    die Durchführung nicht möglich oder nicht erfolgt ist. \\


% ---------------------------------------------------------------------------
% statische Analysetools

\section{Statische Code-Analyse-Tools}

Um eine hohe Code-Qualität sicherzustellen und damit die Wartbarkeit der
Anwendung zu verbessern, wurden statische Analysetools eingesetzt.

\subsection{Checkstyle}

%  A.1.1  Checkstyle
Wir haben das bekannte Tool \glqq Checkstyle\grqq~ verwendet, um die Einhaltung der durch untenstehende Konfiguration festgelegten Richtlinien zum Programmierstil zu überprüfen und durchzusetzen. Diese tragen zu guter Lesbarkeit und geringer Komplexität des Quellcodes bei.

Im Folgenden befindet sich ein Report, der erstellt wurde, als wir begonnen haben, \glqq Checkstyle\grqq~ einzusetzen. Anschließend haben wir alle bereits vorhandenen Warnungen in Commit 5cd67e0aff2b1edd2b42c48fd6b0a36af83a243e \glqq fixes Checkstyle warnings\grqq~ abgearbeitet. Da mit dem Plugin \glqq Checkstyle-IDEA\grqq~ alle Warnungen direkt in unserer Entwicklungsumgebung angezeigt werden, konnten diese ab dann direkt bei der Programmierung behoben werden, sodass der zweite, zum Abschluss des Projektes erstellte, Report leer ist. Ein Screenshot des Plugins mit markierten Fehlern findet sich im Abschnitt nach den Berichten.

Wir haben bei der Checkstyle-Überprüfung auf die Überprüfung verzichtet, ob Felder als private deklariert werden können, da es bei Android eine empfohlene Vorgehensweise ist, einige Felder öffentlich zu deklarieren. Dies sorgt auf mobilen Geräten für mehr Effizienz, da zum Zugreifen auf ein solches Feld keine Methode über einen Call-Stack aufgerufen werden müssen.

\subsubsection{Checkstyle-Konfiguration}

Wir verwenden Checkstyle mit der folgenden Gradle-Konfiguration, um Testdateien nicht der Prüfung zu unterziehen:

\begin{lstlisting}
apply plugin: 'checkstyle'

task checkstyle(type: Checkstyle) {
    source 'src'
    include '**/*.java'
    exclude '**/gen/**','**/*Test.java'
    classpath = files()
}
\end{lstlisting}

Die explizite XML-Konfiguration für Checkstyle selbst sieht wie folgt aus:

\begin{lstlisting}[language=XML]
<?xml version="1.0" encoding="UTF-8" standalone="no"?>

<!DOCTYPE module PUBLIC
    "-//Puppy Crawl//DTD Check Configuration 1.2//EN"
    "http://www.puppycrawl.com/dtds/configuration_1_2.dtd">

<!--
Checkstyle Configuration
Severity: HARD
-->
<module name="Checker">

    <!-- Checks whether files end with a new line.                        -->
    <!-- See http://checkstyle.sourceforge.net/config_misc.html#NewlineAtEndOfFile -->
    <module name="NewlineAtEndOfFile" />

    <!-- Checks that property files contain the same keys.         -->
    <!-- See http://checkstyle.sourceforge.net/config_misc.html#Translation -->
    <module name="Translation" />

    <!-- Checks for Size Violations.                    -->
    <!-- See http://checkstyle.sourceforge.net/config_sizes.html -->
    <module name="FileLength" />

    <!-- Checks for whitespace                               -->
    <!-- See http://checkstyle.sourceforge.net/config_whitespace.html -->
    <module name="FileTabCharacter" />

    <!-- Miscellaneous other checks.                   -->
    <!-- See http://checkstyle.sourceforge.net/config_misc.html -->
    <module name="RegexpSingleline">
        <property name="format" value="\s+$" />
        <property name="minimum" value="0" />
        <property name="maximum" value="0" />
        <property name="message" value="Line has trailing spaces." />
        <property name="severity" value="info" />
    </module>

    <module name="TreeWalker">

        <!-- Checks for Javadoc comments.                     -->
        <!-- See http://checkstyle.sourceforge.net/config_javadoc.html -->
        <module name="JavadocMethod">
            <property name="scope" value="package" />
            <property name="allowMissingParamTags" value="true" />
            <property name="allowMissingThrowsTags" value="true" />
            <property name="allowMissingReturnTag" value="true" />
            <property name="allowThrowsTagsForSubclasses" value="true" />
            <property name="allowUndeclaredRTE" value="true" />
            <property name="allowMissingPropertyJavadoc" value="true" />
            <property name="severity" value="ignore" />
        </module>
        <module name="JavadocType">
            <property name="scope" value="package" />
            <property name="severity" value="ignore" />
        </module>
        <module name="JavadocVariable">
            <property name="scope" value="package" />
            <property name="severity" value="ignore" />
        </module>
        <module name="JavadocStyle">
            <property name="checkEmptyJavadoc" value="true" />
            <property name="severity" value="ignore" />
        </module>

        <!-- Checks for Naming Conventions.                  -->
        <!-- See http://checkstyle.sourceforge.net/config_naming.html -->
        <!-- skip ConstantNameCheck, as it is common practice to have mutable static variables in Android -->
        <!-- <module name="ConstantName" /> -->
        <module name="LocalFinalVariableName" />
        <module name="LocalVariableName" />
        <module name="MemberName" />
        <module name="MethodName" />
        <module name="PackageName" />
        <module name="ParameterName" />
        <module name="StaticVariableName" />
        <module name="TypeName" />

        <!-- Checks for imports                              -->
        <!-- See http://checkstyle.sourceforge.net/config_import.html -->
        <module name="AvoidStarImport" />
        <module name="IllegalImport" />
        <!-- defaults to sun.* packages -->
        <module name="RedundantImport" />
        <module name="UnusedImports" />


        <!-- Checks for Size Violations.                    -->
        <!-- See http://checkstyle.sourceforge.net/config_sizes.html -->
        <module name="LineLength">
            <!-- what is a good max value? -->
            <property name="max" value="150" />
            <!-- ignore lines like "$File: //depot/... $" -->
            <property name="ignorePattern" value="\$File.*\$" />
            <property name="severity" value="info" />
        </module>
        <module name="MethodLength" />


        <!-- Checks for whitespace                               -->
        <!-- See http://checkstyle.sourceforge.net/config_whitespace.html -->
        <module name="EmptyForIteratorPad" />
        <module name="GenericWhitespace" />
        <module name="MethodParamPad" />
        <module name="NoWhitespaceAfter" />
        <module name="NoWhitespaceBefore" />
        <module name="OperatorWrap" />
        <module name="ParenPad" />
        <module name="TypecastParenPad" />
        <module name="WhitespaceAfter" />
        <module name="WhitespaceAround" />

        <!-- Modifier Checks                                    -->
        <!-- See http://checkstyle.sourceforge.net/config_modifiers.html -->
        <module name="ModifierOrder" />
        <module name="RedundantModifier" />


        <!-- Checks for blocks.          -->
        <!-- See http://checkstyle.sourceforge.net/config_blocks.html -->
        <module name="AvoidNestedBlocks" />
        <module name="EmptyBlock">
            <property name="option" value="text" />
        </module>
        <module name="LeftCurly" />
        <module name="NeedBraces" />
        <module name="RightCurly" />


        <!-- Checks for common coding problems               -->
        <!-- See http://checkstyle.sourceforge.net/config_coding.html -->
        <module name="EmptyStatement" />
        <module name="EqualsHashCode" />
        <module name="HiddenField">
            <property name="ignoreConstructorParameter" value="true" />
            <property name="ignoreSetter" value="true" />
            <property name="severity" value="warning" />
        </module>
        <module name="IllegalInstantiation" />
        <module name="InnerAssignment" />
        <module name="MagicNumber">
            <property name="severity" value="info" />
            <property name="ignoreNumbers" value="-1, 0, 1, 2, 3, 4, 5, 8, 10, 12, 16, 24, 30, 32, 60, 64, 100, 120, 128, 180, 256, 360, 1000" />
        </module>
        <module name="MissingSwitchDefault">
            <property name="severity" value="ignore" />
        </module>
        <!-- Problem with finding exception types... -->
        <module name="SimplifyBooleanExpression" />
        <module name="SimplifyBooleanReturn" />

        <!-- Checks for class design                         -->
        <!-- See http://checkstyle.sourceforge.net/config_design.html -->
        <module name="FinalClass" />
        <module name="HideUtilityClassConstructor" />
        <module name="InterfaceIsType" />
        <!-- no visibility modifier checks, as it is common practice in Android to have certain
             publicly accesible fields for performace reasons -->
        <!-- <module name="VisibilityModifier" /> -->


        <!-- Miscellaneous other checks.                   -->
        <!-- See http://checkstyle.sourceforge.net/config_misc.html -->
        <module name="ArrayTypeStyle" />
        <module name="UpperEll" />

    </module>

    <!-- Enable suppression comments -->
    <module name="SuppressionCommentFilter">
        <property name="offCommentFormat" value="CHECKSTYLE IGNORE\s+(\S+)" />
        <property name="onCommentFormat" value="CHECKSTYLE END IGNORE\s+(\S+)" />
        <property name="checkFormat" value="$1" />
    </module>
    <module name="SuppressWithNearbyCommentFilter">
        <property name="commentFormat" value="SUPPRESS CHECKSTYLE (\w+)" />
        <property name="checkFormat" value="$1" />
        <property name="influenceFormat" value="1" />
    </module>

</module>
\end{lstlisting}


\includepdf[pages=1,offset=-0.8cm 0,scale=.8,pagecommand=\subsubsection{Initialer Checkstyle-Report vom 4. Juli 2015}]{anhang/partials/checkstyle-1.pdf}
\includepdf[pages=2-,offset=-0.8cm 0,scale=.8,pagecommand={}]{anhang/partials/checkstyle-1.pdf}

\includepdf[pages=1,offset=-0.8cm 0,scale=.8,pagecommand=\subsubsection{Finaler Checkstyle-Report vom 21. September 2015}]{anhang/partials/checkstyle-2.pdf}
\includepdf[pages=2-,offset=-0.8cm 0,scale=.8,pagecommand={}]{anhang/partials/checkstyle-2.pdf}

\subsubsection{Screenshot von Checkstyle-IDEA}

In diesem Screenshot ist sichtbar, wie Checkstyle direkt und insbesondere vor einem Commit Fehler markiert. Hier wird angemerkt, dass zwischen der Typumwandlung \glqq (Peer)\grqq~ und dem Feld \glqq other\grqq~ ein Leerzeichen fehlt, sowie dass die if-else-Konstruktion keine geschweiften Klammern verwendet.

Durch die auffällige rote Markierung kann die Checkstyle-Warnung nicht übersehen werden.

\includegraphics[width=17.5cm]{belege/checkstyle/checkstyle-idea-screenshot.png}


\clearpage
\subsection{Android Lint}

Zusätzlich zu Checkstyle haben wir das von Google entwickelte und empfohlene \glqq Android Lint\grqq~ genutzt. Dieses Tool ist speziell für Android optimiert und erzeugt im Gegensatz zu Checkstyle weniger Warnungen bezüglich syntaktischer Programmierrichtlinien, sondern vielmehr Empfehlungen zur Verbesserung der Performance, Usability, Accessibility und Korrektheit.

Dadurch verbessert sich einerseits die Wartbarkeit, denn durch das Befolgen der im Android-Umfeld üblichen Konventionen und die Verwendung von allgemein bekannten und anerkannten Entwurfsmustern, ist der Code zukünftigen Entwicklern leichter verständlich.

Im folgenden Abschnitt befindet sich wieder ein Report, der erstellt wurde, als wir begonnen haben, \glqq Android Lint\grqq~ einzusetzen. Dann wurden initial alle Warnungen und Empfehlungen abgearbeitet (Commit 1f4733747f7a98d89e7832937d94a9ddcbdc2bd5 \glqq fixes Android Lint errors and warnings\grqq). Die Warnungen und Empfehlungen werden direkt in der von uns verwendeten Entwicklungsumgebung Android Studio angezeigt, sodass diese anschließend während der Programmierung direkt umgesetzt werden konnten. Daher ist der zweite Report, der zum Abschluss des Projekts erstellt wurde, leer.

Nach den Berichten findet sich ein Screenshot, der markierte Android Lint-Warnungen anzeigt.


\includepdf[pages=1,offset=-0.8cm 0,scale=.8,pagecommand=\subsubsection{Initialer Android Lint Report}]{anhang/partials/lint-results-1.pdf}
\includepdf[pages=2-,offset=-0.8cm 0,scale=.8,pagecommand={}]{anhang/partials/lint-results-1.pdf}

\includepdf[pages=1,offset=-0.8cm 0,scale=.8,pagecommand=\subsubsection{Finaler Android Lint Report}]{anhang/partials/lint-results-2.pdf}

\subsubsection{Screenshot von Android Lint in Android Studio}

In diesem Screenshot ist ersichtlich, wie Android Lint anmerkt, statt \glqq paddingRight\grqq~ \glqq paddingEnd\grqq~ zu verwenden. Dies ist ein Beispiel für Verbesserungen der Accessibility für Android Apps. Weiterhin angemerkt wird, dass die beiden Konstruktionen aus jeweils einem \glqq LinearLayout\grqq~ und einem \glqq ImageView\grqq~ zu einem \glqq Compound Layout\grqq~ zusammengefasst werden können.

Durch die gelbe Hinterlegung in Android Studio, welche bereits beim Tippen des Quellcodes angezeigt wird, können Android Lint Warnungen kaum übersehen werden.

\includegraphics[width=17.5cm]{belege/lint/android-lint-screenshot.png}


\clearpage
\subsection{FindBugs}


Um häufig auftretende Fehler zu vermeiden, haben wir \glqq FindBugs\grqq~ eingesetzt. Im Folgenden befindet sich ein Report, der erstellt wurde, als wir mit der Verwendung von FindBugs begonnen aben. Anschließend wurden alle Fehler und Warnungen in Commit e30930731deb28b4af37f4d820a163f48dcf3ff4 \glqq fixes FindBugs errors\grqq~ behoben. Da wir das Plugin FindBugs-IDEA direkt in unsere Entwicklungsumgebung integriert haben und Fehler so direkt markiert wurden, konnten Fehler ab dann direkt bei der Entwicklung vermieden werden. Bei Projektende sind keine FindBugs-Warnungen vorhanden, es kann aber kein leerer Bericht exportiert werden, da das in FindBugs-IDEA nicht vorgesehen ist.

Nach den Berichten findet sich auch für FindBugs ein Screenshot, welcher \glqq FindBugs-IDEA\grqq~ in Aktion und insbesondere die nahtlose Integration in Android Studio zeigt.

\subsubsection{FindBugs-Konfiguration}

Wir verwenden FindBugs mit folgender Gradle-Konfiguration:

\begin{lstlisting}
apply plugin: 'findbugs'

findbugs {
    excludeFilter = file("$rootProject.projectDir/config/findbugs/excludeFilter.xml")
}
\end{lstlisting}

Unsere exclude-Konfiguration sieht wie folgt aus:

\begin{lstlisting}
<?xml version="1.0" encoding="UTF-8"?>
<FindBugsFilter>
    <Match>
        <!-- ignore all issues in resource generation -->
        <Class name="~.*\.R\$.*"/>
    </Match>
    <Match>
        <Class name="~.*\.Manifest\$.*"/>
    </Match>
</FindBugsFilter>
\end{lstlisting}

\includepdf[pages=1,offset=-0.8cm 0,scale=.8,pagecommand=\subsubsection{Initialer FindBugs-Report}]{anhang/partials/findbugs-1.pdf}
\includepdf[pages=2-,offset=-0.8cm 0,scale=.8,pagecommand={}]{anhang/partials/findbugs-1.pdf}

\subsubsection{Finaler FindBugs-Report}

Es sind keine FindBugs-Fehler vorhanden, davon kann allerdings kein Bericht erstellt werden. Daher folgt hier ein Screenshot von Android Studio als Beleg.

\includegraphics[width=17.5cm]{belege/findbugs/findbugs-no-warnings-screenshot.png}

\subsubsection{Screenshot von FindBugs-IDEA}

Im gezeigten Screenshot ist sichtbar, dass FindBugs eine Stelle im Code markiert hat, an der möglicherweise eine NullPointerException auftreten könnte. Weitere gefundene Fehler sind in der FindBugs-Ansicht unten am Bildschirm erkennbar.

Da betroffene Stellen orange unterstrichen werden und mit einer Fehlermarkierung am Rand angezeigt werden, sind sie direkt bei der Entwicklung der Anwendung ersichtlich und können umgehend behoben werden.

\includegraphics[width=17.5cm]{belege/findbugs/findbugs-idea-screenshot.png}


\clearpage

% ---------------------------------------------------------------------------
% Code Reviews (Einleitung)

\section{Code Reviews}

Zur Sicherstellung der Wartbarkeit durch hohe Code-Qualität wurde bei jedem
Sprint-Treffen mit dem Auftraggeber ein Code Review durchgeführt.

% ---------------------------------------------------------------------------
% Code Reviews: Checkliste

\subsection{Checkliste}

\documentclass[accentcolor=tud2d,12pt,paper=a4,colorbacktitle]{tudexercise}

\usepackage[utf8]{inputenc}
\usepackage{ngerman}
\usepackage{enumerate}

\title{Code Review}
\subtitle{BonfireChat}
\subsubtitle{Bachelorpraktikum 2015}

\begin{document}

  \maketitle

  \begin{enumerate}[ 1.]
    \item Is the functionality correct?
    \item Are the classes named suitably?
    \item Are the functions named suitably?
    \item How's the datastructure being used? Is it the correct DS or it needs improvement?
    \item If it contain functions that can be reused later then are there Utils created for them?
    \item Can it use already available Util functions?
    \item Are all the input arguments being validated?
    \item How will the functionalities affect the performance of the app - time and memory?
    \item Is it absolutely necessary to run it on UI thread or would a background thread suffice?
    \item Are all the fail points handled?
    \item Does it degrade gracefully in case of unknown failures?
    \item Are the app-doing level standards being followed?
    \item Are the operations thread safe?
    \item What pieces of the component can be executed in parallel?
    \item Are the layouts suitable for all the screen dimensions?
    \item What classes would become obsolete after this implementation?
  \end{enumerate}

\end{document}


% ---------------------------------------------------------------------------
% Code Reviews: Ergebnisse
\input{anhang/partials/codereviews.tex}


\clearpage
\section{Dokumentation}

Um die Wartung der Anwendung in Zukunft zu erleichtern, wurden die technischen Grundlagen
sowie alle wichtigen Designentscheidungen in einem Dokument festgehalten.

\input{anhang/partials/dokumentation.tex}



% ---------------------------------------------------------------------------
% Automatisierte Tests: (Einleitung)

\clearpage
\section{Automatisierte Tests}

Es wurden JUnit-Tests für die automatisiert testbaren Codeteile durchgeführt.
Im Folgenden findet sich zunächst eine Liste aller Tests sowie der Testergebnisse,
außerdem das Ergebnis der Test-Coverage-Analyse, aufgeschlüsselt nach Packages.
\\\\
Wie bereits im QS-Dokument beschrieben, war ein Testen des UI- und Netzwerkcodes
(Packages bonfirechat.network und bonfirechat.ui)
mit dem JUnit-Framework nicht möglich, da die Android-Klassenbibliothek dort nicht
zur Verfügung steht. Das gleiche gilt für die Datenbankklasse, da diese von einer
Android-internen Klasse erbt. Daher konnte auch die Klasse bonfire.data.BonfireData
nicht mit automatisierten Tests versehen werden.
\\\\
Zusammenfassend gilt, dass wir automatisierte Tests für alle Methoden geschrieben haben,
die
\begin{itemize}
\item keine Klassen der Android-Klassenbibliothek verwenden,
\item Klassen der Android-Klassenbibliothek nur als Parameter übergeben bekommen,
sodass wir stattdessen mit Mockito erstellte Mock-Objekte übergeben können, oder
\item mit vertretbarem Aufwand angepasst werden konnten, sodass der vorhergehende Punkt zutrifft.
\end{itemize}

% ---------------------------------------------------------------------------
% Automatisierte Tests: Javadoc-Testplan

\includepdf[pages=1,scale=.8,pagecommand=\subsection{Testplan}]{anhang/partials/javadoc.pdf}
\includepdf[pages=2-,scale=.8,pagecommand={}]{anhang/partials/javadoc.pdf}

% ---------------------------------------------------------------------------
% Automatisierte Tests: Ergebnisse

\includepdf[pages=1,offset=-0.8cm 0,scale=.8,pagecommand=\subsection{Ergebnisse}]{anhang/partials/junit.pdf}
\includepdf[pages=2-,offset=-0.8cm 0,scale=.8,pagecommand={}]{anhang/partials/junit.pdf}


% ---------------------------------------------------------------------------
% Automatisierte Tests: Test Coverage

\subsection{Test Coverage}

Aufgrund der in der Einleitung zu \glqq Automatisierte Tests\grqq~ beschriebenen Probleme mit JUnit und Android können große Teile unseres Quellcodes nicht automatisiert getestet werden. Daher scheint auch die Testabdeckung relativ gering zu sein, es ist jedoch nicht möglich mehr Abdeckung zu erhalten.

\hspace{-1cm}
\includegraphics[width=19.3cm,clip=true,trim=0cm 15cm 0cm 0cm]{anhang/partials/coverage.pdf}

\clearpage

\section{Manuelle Tests der Benutzeroberfläche}

Da uns der Aufwand für die Nutzung eines Instrumented-Test-Frameworks
für das recht einfache User Interface der App unverhältnismäßig
erscheint, werden manuelle Tests anhand des folgenden Testplans
vorgenommen.

\subsection{Testplan}

Dieser Abschnitt beschreibt einen Testplan für manuelles Testen der
drahtlosen Übertragung sowie des Routing-Algorithmus.\\\\

% ----------------------------------------------------------------------

\subsubsection{N-i: Definitionen}\label{i-definitionen}

\paragraph{Allgemeine Ausgangskonfiguration der
Knoten:}\label{allgemeine-ausgangskonfiguration-der-knoten}

Ein Knoten ist, soweit nicht näher beschrieben, ein Android-Gerät, auf
dem die aktuellste Version der App eingerichtet und die initiale
Einrichtung (Eingabe eines Nicknames, dadurch Registrierung des Gerätes
mit Nickname, GCM-ID und PublicKey am Server) abgeschlossen ist.


% ----------------------------------------------------------------------
% II

\clearpage
\subsubsection{N-ii: Testfälle für die allgemeine
Übertragung}\label{ii-testfuxe4lle-fuxfcr-die-allgemeine-uxfcbertragung}

\paragraph{N-ii-1. Test der direkten Übertragung via Flooding /
Bluetooth}\label{test-der-direkten-uxfcbertragung-via-flooding-bluetooth}

\begin{longtable}{p{8cm}p{8.5cm}}
\toprule
Benutzerinteraktion & erwartetes Verhalten der App\tabularnewline
\midrule
\endhead
Auf zwei Knoten A und B, die sich in direkter Bluetooth-Reichweite
befinden, wird zunächst die Ausgangskonfiguration hergestellt,
anschließend werden in den Einstellungen alle Übertragungsverfahren
außer Bluetooth deaktiviert. Die Kontaktdaten von A werden an B
gesendet. Auf B wird eine neue Unterhaltung mit A gestartet. Auf B wird
eine Nachricht an A eingegeben und abgesendet. & B sendet die Nachricht
an alle per Bluetooth sichtbaren Geräte, insbesondere an A. Auf A wird
die Nachricht empfangen und als per Bluetooth empfangen dargestellt. A
sendet ein ACK an B, dieses wird auf B durch ein Häkchen an der
Nachricht sichtbar. Weiterhin erscheint ein Bluetooth-Icon an der
Nachricht, da die Nachricht per Bluetooth zugestellt wurde. Im Dashboard
ist erkennbar, dass die Nachricht mit Routingmodus Flooding (0x01)
versendet wurde, das ACK mit Routingmodus DSR (0x02).\tabularnewline
\bottomrule
\end{longtable}

\paragraph{N-ii-2. Test der direkten Übertragung via DSR /
Bluetooth}\label{test-der-direkten-uxfcbertragung-via-dsr-bluetooth}

\begin{longtable}{p{8cm}p{8.5cm}}
\toprule
Benutzerinteraktion & erwartetes Verhalten der App\tabularnewline
\midrule
\endhead
Nach der erfolgreichen Durchführung von Test 1 wird eine weitere
Nachricht von B an A gesendet, sowie eine Nachricht von A an B. & Nach
Test 1 ist den Geräten A und B der schnellste Pfad zum jeweils anderen
Gerät bekannt. Daher sendet B die Nachricht nicht per Routingmodus
Flooding (0x01), sondern per Dynamic Source Routing (0x02), also unter
der Angabe der gewünschten Übertragungspfades. Daher wird die Nachricht
nur an Gerät A gesendet. Darüber hinaus identisch zu Test
1.\tabularnewline
\bottomrule
\end{longtable}

\paragraph{N-ii-3. Test der direkten Übertragung via Server / Google Cloud
Messaging}\label{test-der-direkten-uxfcbertragung-via-server-google-cloud-messaging}

\begin{longtable}{p{8cm}p{8.5cm}}
\toprule
Benutzerinteraktion & erwartetes Verhalten der App\tabularnewline
\midrule
\endhead
Auf zwei Knoten A und B, die eine funktionierende Internetverbindung
haben, wird zunächst die Ausgangskonfiguration hergestellt, anschließend
werden in den Einstellungen alle Übertragungsverfahren außer ``mobile
Daten / Wifi'' (Übertragung per Cloud) deaktiviert. Die Kontaktdaten von
A werden an B gesendet. Auf B wird eine neue Unterhaltung mit A
gestartet. Auf B wird eine Nachricht an A eingegeben und abgesendet. & B
sendet die Nachricht per HTTP an den Server, welcher sie per GCM an
Gerät A weiterleitet. Auf A wird die Nachricht empfangen und als per
Cloud empfangen dargestellt. A sendet ein ACK an B, dieses wird auf B
durch ein Häkchen an der Nachricht sichtbar. Weiterhin erscheint ein
Cloud-Icon an der Nachricht, da die Nachricht per Cloud zugestellt
wurde. Im Dashboard ist erkennbar, dass die Nachricht mit Routingmodus
Flooding (0x01) versendet wurde, das ACK mit Routingmodus DSR
(0x02).\tabularnewline
\bottomrule
\end{longtable}


% ----------------------------------------------------------------------
% III

\clearpage
\subsubsection{N-iii: Testfälle für Wegfindung per Flooding und einfaches
Routing}\label{iii-testfuxe4lle-fuxfcr-wegfindung-per-flooding-und-einfaches-routing}

In diesen Tests werden folgende Teile des Routing getestet: * Finden des
schnellsten Pfades: Flooding an alle Knoten sowie ACK auf dem Pfad, auf
dem die Nachricht den Empfänger zuerst erreicht (= schnellster Pfad) *
Speichern des schnellsten Pfades zu einem Empfänger * Verwenden des
gespeicherten schnellsten Pfades für künftige Nachrichten

\paragraph{N-iii-1. Bluetooth-Wegfindung - Flooding mit drei
Knoten}\label{bluetooth-wegfindung---flooding-mit-drei-knoten}

\begin{longtable}{p{8cm}p{8.5cm}}
\toprule
Benutzerinteraktion & erwartetes Verhalten der App\tabularnewline
\midrule
\endhead
Auf drei Knoten A, B und C werden alle Übertragungsverfahren außer
Bluetooth deaktiviert. Die Kontaktdetails von C werden an A gesendet.
Die Knoten werden räumlich so angeordnet, dass eine Bluetoothverbindung
zwischen A und B sowie zwischen B und C möglich ist, nicht jedoch
zwischen A und C. Auf A wird eine Unterhaltung mit C begonnen und eine
Nachricht an C gesendet. & Nachricht u. ACK kommen an,
etc.\tabularnewline
\bottomrule
\end{longtable}

\paragraph{N-iii-2. Bluetooth-Wegfindung zum nächsten Knoten mit
Internetverbindung}\label{bluetooth-wegfindung-zum-nuxe4chsten-knoten-mit-internetverbindung}

\begin{longtable}{p{8cm}p{8.5cm}}
\toprule
Benutzerinteraktion & erwartetes Verhalten der App\tabularnewline
\midrule
\endhead
Knoten A, B und C werden wie in Testfall III.1 vorbereitet. Auf Knoten C
wird zusätzlich die Übertragung per Cloud aktiviert. Auf einem weiteren
Knoten D wird nur die Übertragung per Cloud aktiviert. Es wird
sichergestellt, dass C und D mit dem Internet verbunden sind. Die
Kontaktdetails von D werden an A gesendet. A beginnt eine Unterhaltung
mit D und sendet die Nachricht ``alpha'' an D. Nach Erhalt der Nachricht
``alpha'' sendet A eine weitere Nachricht ``beta''. & Die Nachricht
``alpha'' kommt per Flooding über B, C und Server bei D an, ACK ``für
alpha'' geht auf direktem Pfad (D-Server-C-B-A) per DSR zurück an A. Die
Nachricht ``beta'' wird auf direktem Pfad (A-B-C-Server-D) per DSR an D
gesendet, ACK ``für beta'' wie ACK ``für alpha''. Die empfangenen und
weitergeleiteten Nachrichten und ihre Pfade sind entsprechend im
Dashboard ersichtlich.\tabularnewline
\bottomrule
\end{longtable}

\paragraph{N-iii-3.}\label{section}

TODO Test beschreiben




% ------------------------------------------------------------------------
% IV

\clearpage
\subsubsection{N-iv: Testfälle für sich verändernde
Routen}\label{iv-testfuxe4lle-fuxfcr-sich-veruxe4ndernde-routen}

In diesen Tests wird überprüft, ob der Routingalgorithmus bei
ausfallenden Pfaden korrekt reagiert.

\paragraph{\texorpdfstring{N-iv-1. ``Abreißende''
Bluetooth-Verbindung}{N-iv-1. Abreißende Bluetooth-Verbindung}}\label{abreiuxdfende-bluetooth-verbindung}

\begin{longtable}{p{8cm}p{8.5cm}}
\toprule
Benutzerinteraktion & erwartetes Verhalten der App\tabularnewline
\midrule
\endhead
Auf drei Knoten A wird nur Bluetooth aktiviert, auf B wird GCM und
Bluetooth aktiviert. Von A wird eine Nachricht ``alpha'' an B gesendet.
Nachdem diese zugestellt und acknowledged wurde, wird eine weitere
Nachricht ``beta'' von A and B gesendet. Danach wird A räumlich so weit
vom B entfernt, dass keine Bluetooth-Übertragung mehr möglich ist.
Weiterhin wird auf A die Übertragung per GCM aktiviert. Anschließend
wird eine weitere Nachricht ``gamma'' von A and B gesendet. & Die
Nachricht ``alpha'' wird per Flooding gesendet, B sendet per DSR ein ACK
``für alpha'' zurück. Danach hat A den Pfad zu B gespeichert und die
Nachricht ``beta'' wird direkt per DSR an B gesendet. Die Nachricht
``gamma'' wird auch versucht per DSR direkt via Bluetooth an B zu
senden. Da dies fehlschlägt, wird beim Retry nach 20 Sekunden versucht,
``gamma'' per Flooding zuzustellen. Dies erfolgt dann über GCM. Das ACK
``für gamma'' von B an A erfolgt dann wiederum per DSR.\tabularnewline
\bottomrule
\end{longtable}



% ------------------------------------------------------------------------
% V

\clearpage
\subsubsection{N-v: Testfälle für Störeinflüsse im
Netzwerk}\label{v-testfuxe4lle-fuxfcr-stuxf6reinfluxfcsse-im-netzwerk}

Hier soll überprüft werden, wie sich die App bei äußeren Störeinflüssen
auf Netzwerkebene, also zum Beispiel während einer während der
Datenübertragung abreißenden Verbindung, verhält. Es soll sichergestellt
sein, dass stets angemessen reagiert wird und sowohl die App nicht
abstürzt als auch der Benutzer sinnvolle Fehlermeldungen erhält.

Da dies teilweise sporadische Fehler sind, welche sich schwer
reproduzieren lassen, ist die Aussagekraft der folgenden Tests leider
nur bedingt gegeben.

\paragraph{\texorpdfstring{N-v-1. ``Socket wird unerwartet während der
Verbindung
geschlossen''}{N-v-1. Socket wird unerwartet während der Verbindung geschlossen}}\label{socket-wird-unerwartet-wuxe4hrend-der-verbindung-geschlossen}

\begin{longtable}{p{8cm}p{8.5cm}}
\toprule
Benutzerinteraktion & erwartetes Verhalten der App\tabularnewline
\midrule
\endhead
\bottomrule
\end{longtable}

\paragraph{N-v-2. Benutzer schaltet Bluetooth im Telefon
aus}\label{benutzer-schaltet-bluetooth-im-telefon-aus}

\begin{longtable}{p{8cm}p{8.5cm}}
\toprule
Benutzerinteraktion & erwartetes Verhalten der App\tabularnewline
\midrule
\endhead
\bottomrule
\end{longtable}


\input{anhang/partials/manuelle-tests-ui-reports.tex}


\clearpage
\section{Manueller Test für Routing und Datenübertragungs-Protokolle}

Ebenso wie bei den manuellen Tests der Benutzeroberfläche ist es kaum möglich, das Netzwerkverhalten der Anwendung automatisiert zu testen, da dafür physikalische Eigenschaften der Funkübertragung simuliert werden müssten. Wir haben daher manuelle Tests anhand des folgenden Testplans vorgenommen.

\subsection{Testplan}

Dieser Abschnitt beschreibt einen Testplan für manuelles Testen der
drahtlosen Übertragung sowie des Routing-Algorithmus.\\\\

% ----------------------------------------------------------------------

\subsubsection{N-i: Definitionen}\label{i-definitionen}

\paragraph{Allgemeine Ausgangskonfiguration der
Knoten:}\label{allgemeine-ausgangskonfiguration-der-knoten}

Ein Knoten ist, soweit nicht näher beschrieben, ein Android-Gerät, auf
dem die aktuellste Version der App eingerichtet und die initiale
Einrichtung (Eingabe eines Nicknames, dadurch Registrierung des Gerätes
mit Nickname, GCM-ID und PublicKey am Server) abgeschlossen ist.


% ----------------------------------------------------------------------
% II

\clearpage
\subsubsection{N-ii: Testfälle für die allgemeine
Übertragung}\label{ii-testfuxe4lle-fuxfcr-die-allgemeine-uxfcbertragung}

\paragraph{N-ii-1. Test der direkten Übertragung via Flooding /
Bluetooth}\label{test-der-direkten-uxfcbertragung-via-flooding-bluetooth}

\begin{longtable}{p{8cm}p{8.5cm}}
\toprule
Benutzerinteraktion & erwartetes Verhalten der App\tabularnewline
\midrule
\endhead
Auf zwei Knoten A und B, die sich in direkter Bluetooth-Reichweite
befinden, wird zunächst die Ausgangskonfiguration hergestellt,
anschließend werden in den Einstellungen alle Übertragungsverfahren
außer Bluetooth deaktiviert. Die Kontaktdaten von A werden an B
gesendet. Auf B wird eine neue Unterhaltung mit A gestartet. Auf B wird
eine Nachricht an A eingegeben und abgesendet. & B sendet die Nachricht
an alle per Bluetooth sichtbaren Geräte, insbesondere an A. Auf A wird
die Nachricht empfangen und als per Bluetooth empfangen dargestellt. A
sendet ein ACK an B, dieses wird auf B durch ein Häkchen an der
Nachricht sichtbar. Weiterhin erscheint ein Bluetooth-Icon an der
Nachricht, da die Nachricht per Bluetooth zugestellt wurde. Im Dashboard
ist erkennbar, dass die Nachricht mit Routingmodus Flooding (0x01)
versendet wurde, das ACK mit Routingmodus DSR (0x02).\tabularnewline
\bottomrule
\end{longtable}

\paragraph{N-ii-2. Test der direkten Übertragung via DSR /
Bluetooth}\label{test-der-direkten-uxfcbertragung-via-dsr-bluetooth}

\begin{longtable}{p{8cm}p{8.5cm}}
\toprule
Benutzerinteraktion & erwartetes Verhalten der App\tabularnewline
\midrule
\endhead
Nach der erfolgreichen Durchführung von Test 1 wird eine weitere
Nachricht von B an A gesendet, sowie eine Nachricht von A an B. & Nach
Test 1 ist den Geräten A und B der schnellste Pfad zum jeweils anderen
Gerät bekannt. Daher sendet B die Nachricht nicht per Routingmodus
Flooding (0x01), sondern per Dynamic Source Routing (0x02), also unter
der Angabe der gewünschten Übertragungspfades. Daher wird die Nachricht
nur an Gerät A gesendet. Darüber hinaus identisch zu Test
1.\tabularnewline
\bottomrule
\end{longtable}

\paragraph{N-ii-3. Test der direkten Übertragung via Server / Google Cloud
Messaging}\label{test-der-direkten-uxfcbertragung-via-server-google-cloud-messaging}

\begin{longtable}{p{8cm}p{8.5cm}}
\toprule
Benutzerinteraktion & erwartetes Verhalten der App\tabularnewline
\midrule
\endhead
Auf zwei Knoten A und B, die eine funktionierende Internetverbindung
haben, wird zunächst die Ausgangskonfiguration hergestellt, anschließend
werden in den Einstellungen alle Übertragungsverfahren außer ``mobile
Daten / Wifi'' (Übertragung per Cloud) deaktiviert. Die Kontaktdaten von
A werden an B gesendet. Auf B wird eine neue Unterhaltung mit A
gestartet. Auf B wird eine Nachricht an A eingegeben und abgesendet. & B
sendet die Nachricht per HTTP an den Server, welcher sie per GCM an
Gerät A weiterleitet. Auf A wird die Nachricht empfangen und als per
Cloud empfangen dargestellt. A sendet ein ACK an B, dieses wird auf B
durch ein Häkchen an der Nachricht sichtbar. Weiterhin erscheint ein
Cloud-Icon an der Nachricht, da die Nachricht per Cloud zugestellt
wurde. Im Dashboard ist erkennbar, dass die Nachricht mit Routingmodus
Flooding (0x01) versendet wurde, das ACK mit Routingmodus DSR
(0x02).\tabularnewline
\bottomrule
\end{longtable}


% ----------------------------------------------------------------------
% III

\clearpage
\subsubsection{N-iii: Testfälle für Wegfindung per Flooding und einfaches
Routing}\label{iii-testfuxe4lle-fuxfcr-wegfindung-per-flooding-und-einfaches-routing}

In diesen Tests werden folgende Teile des Routing getestet: * Finden des
schnellsten Pfades: Flooding an alle Knoten sowie ACK auf dem Pfad, auf
dem die Nachricht den Empfänger zuerst erreicht (= schnellster Pfad) *
Speichern des schnellsten Pfades zu einem Empfänger * Verwenden des
gespeicherten schnellsten Pfades für künftige Nachrichten

\paragraph{N-iii-1. Bluetooth-Wegfindung - Flooding mit drei
Knoten}\label{bluetooth-wegfindung---flooding-mit-drei-knoten}

\begin{longtable}{p{8cm}p{8.5cm}}
\toprule
Benutzerinteraktion & erwartetes Verhalten der App\tabularnewline
\midrule
\endhead
Auf drei Knoten A, B und C werden alle Übertragungsverfahren außer
Bluetooth deaktiviert. Die Kontaktdetails von C werden an A gesendet.
Die Knoten werden räumlich so angeordnet, dass eine Bluetoothverbindung
zwischen A und B sowie zwischen B und C möglich ist, nicht jedoch
zwischen A und C. Auf A wird eine Unterhaltung mit C begonnen und eine
Nachricht an C gesendet. & Nachricht u. ACK kommen an,
etc.\tabularnewline
\bottomrule
\end{longtable}

\paragraph{N-iii-2. Bluetooth-Wegfindung zum nächsten Knoten mit
Internetverbindung}\label{bluetooth-wegfindung-zum-nuxe4chsten-knoten-mit-internetverbindung}

\begin{longtable}{p{8cm}p{8.5cm}}
\toprule
Benutzerinteraktion & erwartetes Verhalten der App\tabularnewline
\midrule
\endhead
Knoten A, B und C werden wie in Testfall III.1 vorbereitet. Auf Knoten C
wird zusätzlich die Übertragung per Cloud aktiviert. Auf einem weiteren
Knoten D wird nur die Übertragung per Cloud aktiviert. Es wird
sichergestellt, dass C und D mit dem Internet verbunden sind. Die
Kontaktdetails von D werden an A gesendet. A beginnt eine Unterhaltung
mit D und sendet die Nachricht ``alpha'' an D. Nach Erhalt der Nachricht
``alpha'' sendet A eine weitere Nachricht ``beta''. & Die Nachricht
``alpha'' kommt per Flooding über B, C und Server bei D an, ACK ``für
alpha'' geht auf direktem Pfad (D-Server-C-B-A) per DSR zurück an A. Die
Nachricht ``beta'' wird auf direktem Pfad (A-B-C-Server-D) per DSR an D
gesendet, ACK ``für beta'' wie ACK ``für alpha''. Die empfangenen und
weitergeleiteten Nachrichten und ihre Pfade sind entsprechend im
Dashboard ersichtlich.\tabularnewline
\bottomrule
\end{longtable}

\paragraph{N-iii-3.}\label{section}

TODO Test beschreiben




% ------------------------------------------------------------------------
% IV

\clearpage
\subsubsection{N-iv: Testfälle für sich verändernde
Routen}\label{iv-testfuxe4lle-fuxfcr-sich-veruxe4ndernde-routen}

In diesen Tests wird überprüft, ob der Routingalgorithmus bei
ausfallenden Pfaden korrekt reagiert.

\paragraph{\texorpdfstring{N-iv-1. ``Abreißende''
Bluetooth-Verbindung}{N-iv-1. Abreißende Bluetooth-Verbindung}}\label{abreiuxdfende-bluetooth-verbindung}

\begin{longtable}{p{8cm}p{8.5cm}}
\toprule
Benutzerinteraktion & erwartetes Verhalten der App\tabularnewline
\midrule
\endhead
Auf drei Knoten A wird nur Bluetooth aktiviert, auf B wird GCM und
Bluetooth aktiviert. Von A wird eine Nachricht ``alpha'' an B gesendet.
Nachdem diese zugestellt und acknowledged wurde, wird eine weitere
Nachricht ``beta'' von A and B gesendet. Danach wird A räumlich so weit
vom B entfernt, dass keine Bluetooth-Übertragung mehr möglich ist.
Weiterhin wird auf A die Übertragung per GCM aktiviert. Anschließend
wird eine weitere Nachricht ``gamma'' von A and B gesendet. & Die
Nachricht ``alpha'' wird per Flooding gesendet, B sendet per DSR ein ACK
``für alpha'' zurück. Danach hat A den Pfad zu B gespeichert und die
Nachricht ``beta'' wird direkt per DSR an B gesendet. Die Nachricht
``gamma'' wird auch versucht per DSR direkt via Bluetooth an B zu
senden. Da dies fehlschlägt, wird beim Retry nach 20 Sekunden versucht,
``gamma'' per Flooding zuzustellen. Dies erfolgt dann über GCM. Das ACK
``für gamma'' von B an A erfolgt dann wiederum per DSR.\tabularnewline
\bottomrule
\end{longtable}



% ------------------------------------------------------------------------
% V

\clearpage
\subsubsection{N-v: Testfälle für Störeinflüsse im
Netzwerk}\label{v-testfuxe4lle-fuxfcr-stuxf6reinfluxfcsse-im-netzwerk}

Hier soll überprüft werden, wie sich die App bei äußeren Störeinflüssen
auf Netzwerkebene, also zum Beispiel während einer während der
Datenübertragung abreißenden Verbindung, verhält. Es soll sichergestellt
sein, dass stets angemessen reagiert wird und sowohl die App nicht
abstürzt als auch der Benutzer sinnvolle Fehlermeldungen erhält.

Da dies teilweise sporadische Fehler sind, welche sich schwer
reproduzieren lassen, ist die Aussagekraft der folgenden Tests leider
nur bedingt gegeben.

\paragraph{\texorpdfstring{N-v-1. ``Socket wird unerwartet während der
Verbindung
geschlossen''}{N-v-1. Socket wird unerwartet während der Verbindung geschlossen}}\label{socket-wird-unerwartet-wuxe4hrend-der-verbindung-geschlossen}

\begin{longtable}{p{8cm}p{8.5cm}}
\toprule
Benutzerinteraktion & erwartetes Verhalten der App\tabularnewline
\midrule
\endhead
\bottomrule
\end{longtable}

\paragraph{N-v-2. Benutzer schaltet Bluetooth im Telefon
aus}\label{benutzer-schaltet-bluetooth-im-telefon-aus}

\begin{longtable}{p{8cm}p{8.5cm}}
\toprule
Benutzerinteraktion & erwartetes Verhalten der App\tabularnewline
\midrule
\endhead
\bottomrule
\end{longtable}


\subsection{Durchführung des manuellen Netzwerk-Testplans vom 30. August 2015}

\textbf{Durchgeführt von:} Matthias Hofmann

\textbf{Änderungen vorgenommen in Commit:} keine Änderungen nötig

\paragraph{Teil N-ii Testfälle für die allgemeine Übertragung}

\paragraph{Test N-ii.1 Test der direkten Übertragung via Flooding / Bluetooth}

Der Test war erfolgreich.

\includegraphics[trim=0 180 0 0,clip,scale=0.8]{belege/manuelle-tests/netzwerk/Dashboardauszuege/Netzwerktest_II-1.pdf}
\clearpage

\paragraph{Test N-ii.2 Test der direkten Übertragung via DSR / Bluetooth}

Der Test war erfolgreich.

\includegraphics[trim=0 120 0 0,clip,scale=0.8]{belege/manuelle-tests/netzwerk/Dashboardauszuege/Netzwerktest_II-2a.pdf}
\clearpage
\includegraphics[trim=0 120 0 0,clip,scale=0.8]{belege/manuelle-tests/netzwerk/Dashboardauszuege/Netzwerktest_II-2b.pdf}
\clearpage

\paragraph{Test N-ii.3 Test der direkten Übertragung via Server / Google Cloud Messaging}

Der Test war erfolgreich.

\includegraphics[trim=0 120 0 0,clip,scale=0.8]{belege/manuelle-tests/netzwerk/Dashboardauszuege/Netzwerktest_II-3.pdf}
\clearpage


\paragraph{Teil N-iii Testfälle für Wegfindung per Flooding und einfaches Routing}

\paragraph{Test N-iii.1 Bluetooth-Wegfindung - Flooding mit drei Knoten}

Der Test war erfolgreich.

\includegraphics[trim=0 100 0 0,clip,scale=0.8]{belege/manuelle-tests/netzwerk/Dashboardauszuege/Netzwerktest_III-1.pdf}
\clearpage

\paragraph{Test N-iii-2 Bluetooth-Wegfindung zum nächsten Knoten mit Internetverbindung}

Der Test war erfolgreich.

\includegraphics[trim=0 120 0 0,clip,scale=0.8]{belege/manuelle-tests/netzwerk/Dashboardauszuege/Netzwerktest_III-2.pdf}
\clearpage


\paragraph{Teil N-iv Testfälle für sich verändernde Routen}

\paragraph{Test N-iv.1 \glqq Abreißende\grqq~ Bluetooth-Verbindung}

Der Test war erfolgreich.

\includegraphics[trim=0 120 0 0,clip,scale=0.8]{belege/manuelle-tests/netzwerk/Dashboardauszuege/Netzwerktest_IV-1a.pdf}
\clearpage
\includegraphics[trim=0 90 0 0,clip,scale=0.8]{belege/manuelle-tests/netzwerk/Dashboardauszuege/Netzwerktest_IV-1b.pdf}
\clearpage
\includegraphics[trim=0 30 0 0,clip,scale=0.8]{belege/manuelle-tests/netzwerk/Dashboardauszuege/Netzwerktest_IV-1c.pdf}
\clearpage



\clearpage
\section{Großtests}

Um das Verhalten der App bei einem tatsächlichen Einsatz mit vielen Benutzern und vielen Geräten zu simulieren, haben wir Großtests nach dem folgenden Testplan durchgeführt.

\subsection{Testplan}

\paragraph{Allgemeine Ausgangskonfiguration der Knoten}

Ein Knoten ist, soweit nicht näher beschrieben, ein Android-Gerät, auf dem die aktuellste Version der App eingerichtet und die initiale Einrichtung (Eingabe eines Nicknames, dadurch Registrierung des Gerätes mit Nickname, GCM-ID und PublicKey am Server) abgeschlossen ist.

\paragraph{Eigenschaften}

\begin{itemize}
  \tightlist
  \item Geringe Distanz: Die Knoten stehen nahe zu ihren Nachbarn (1m)
  \item Mittlere Distanz: Die Knoten stehen etwas entfernt zu ihren Nachbarn (5m)
  \item Große Distanz: Die Knoten stehen weit entfernt zu ihren Nachbarn (10m)
\end{itemize}

\paragraph{Testfälle für Statistik Erhebungen}

\textbf{Testaufbau}\\
Wir führen einen Test mit zehn Handys durch. Es wird nur Bluetooth getestet. Alle Handys haben dauerhaft eine Verbindung zum Internet, um ständige Nachrichtenverläufe an unser Dashboard zu senden, damit wir den Test auswerten können. Die Handys werden kugelförmig positioniert und während des Tests nicht bewegt. Es werden viele Nachrichten verschickt (etwa eine Nachricht alle 2 Sekunden). Die Handys können jedoch keine Nachrichten über das Internet versenden, da dies in der App deaktiviert wird. Die Nachrichten werden alle per Hand verschickt, kleine Abweichungen bei Distanz oder Sendegeschwindigkeit werden hingenommen. Jeder Test läuft 5 Minuten.

\textbf{Test 1}\\
Die einzelnen Handys haben eine geringe Distanz zueinander.

\textbf{Test 2}\\
Die einzelnen Handys haben eine mittlere Distanz zueinander.

\textbf{Test 3}\\
Die einzelnen Handys haben eine große Distanz zueinander.

\clearpage

\input{belege/grosstests/grosstest-report-1.tex}
\clearpage

\subsection{Durchführung des Großtests 2}

Nach dem letzten Test wurde der Code so erweitert, dass Bluetooth nur
noch 4 Verbindungen gleichzeitig aufbaut und in regelmäßigen Abständen
alte Verbindungen fallen lässt. Das hindert den Bluetooth Adapter zu
schnell zu überlasten.

Außerdem wurden weitere Kommilitonen darum gebeten am Test teilzunehmen.
Insgesamt wurde der Test von 9 Personen mit 14 Handys ausgeführt.

Die Benennung der Handys wurde den einzelnen Personen überlassen. Es
wurde dafür gesorgt, dass alle Handys den Kontakt der anderen Handys
kennen.


\subsubsection{Ergebnisse Test 1}

Wir haben auf 167 Nachrichten 8 erfolgreiche Fälle und 159 Errors.

\paragraph{Folgende Beispiele veranschaulichen Nachrichten, die beim
Empfänger angekommen sind:}

\includegraphics[width=1.0\textwidth]{belege/grosstests/Bilder/Grosstest2/Test1Erfolg2.jpg}
1. mnex2 sendet eine Nachricht an F2. Diese kommt erst nach einer
Retransmission an. Das Acknowledgement kommt auch erst beim 2. Versuch
an.

\includegraphics[width=1.0\textwidth]{belege/grosstests/Bilder/Grosstest2/Test1Erfolg1.jpg}

2. Hier sendet a2 eine Nachricht an F2. Da F2 ein Nachbar ist, kommt die
Nachricht direkt an. A2 versucht gleichzeitig auch noch einen weiteren
Weg, der jedoch nach einem Hop verloren geht.

\includegraphics[width=1.0\textwidth]{belege/grosstests/Bilder/Grosstest2/Test1Erfolg3.jpg}

3. In dieser Nachricht sendet 3 an 2. Die Nachricht wird erfolgreich über 1
geschickt. 1 sendet die Nachricht gleichzeitig auch noch an a2.

\paragraph{Folgende Beispiele veranschaulichen Nachrichten, die nicht beim
Empfänger angekommen sind:}

\includegraphics[width=1.0\textwidth]{belege/grosstests/Bilder/Grosstest2/Test1Misserfolg2.jpg}

1. Bei diesem Test kam es häufig zur Situation, dass Handys überhaupt keine
Verbindung mit anderen Geräten herstellen konnten. Dies kann daran
liegen, dass die Geräte in der näheren Umgebung bereits 4 Verbindungen
offen hatten, oder dessen Bluetooth durch frühere Transmissions
überlastet war.

\includegraphics[width=1.0\textwidth]{belege/grosstests/Bilder/Grosstest2/Test1Misserfolg1.jpg}

2. Hier sendet a2 eine Nachricht, die jedoch nur über einen Hop
weitergeleitet wird und deshalb nicht ankommt.

\includegraphics[width=1.0\textwidth]{belege/grosstests/Bilder/Grosstest2/Test1Misserfolg3.jpg}

3. Auch hier sendet a2 eine Nachricht, die über die Hops F2, Nightfire und
2 läuft, dort jedoch verloren geht.

\paragraph{Schlussfolgerung aus dem Test:}

Die Einschränkung auf 4 Verbindungen pro Bluetooth-Gerät scheint nicht
erfolgreich gewesen zu sein. Der Bluetoothadapter scheint immer noch
sehr schnell zu überlasten. Die Einschränkung führt außerdem noch dazu,
dass nicht alle Handys in der Umgebung erreicht werden, da bereits 4
Verbindungen bestehen, oder versucht aufgebaut zu werden.\\

Es bestand kein Grund zur Annahme, dass Verbindungen bei diesem Test
nicht zustande gekommen sind, weil ein Handy in den Standby-Modus
gegangen ist.


\subsubsection{Ergebnisse Test 2}

Wir haben auf 227 Nachrichten 20 erfolgreiche Fälle und 7 Errors.

\paragraph{Folgende Beispiele veranschaulichen Nachrichten, die beim Empfänger angekommen sind:}

\includegraphics[width=1.0\textwidth]{belege/grosstests/Bilder/Grosstest2/Test2Erfolg1.jpg}

1. Die Verbindung zwischen mnex2 und Sven bestand über den ganzen Test.
Über diese Verbindung gingen 13 Nachrichten.

\includegraphics[width=1.0\textwidth]{belege/grosstests/Bilder/Grosstest2/Test2Erfolg3.jpg}

2. G3n3siS sendet eine Nachricht an Sven. Diese wird erfolgreich an a2
übermittelt, welches diese dann an Sven weiterleitet. Das
Acknowledgement kommt jedoch nicht mehr zurück.

\includegraphics[width=1.0\textwidth]{belege/grosstests/Bilder/Grosstest2/Test2Erfolg4.jpg}

3. Nightfire sendet eine Nachricht an mnex2.

\paragraph{Folgende Beispiele veranschaulichen Nachrichten, die nicht beim Empfänger angekommen sind:}

\includegraphics[width=1.0\textwidth]{belege/grosstests/Bilder/Grosstest2/Test2Misserfolg1.jpg}

1. Im Verhältnis zu Test 1 gab es noch häufiger die Situation, dass Handys
gar nicht mit anderen Geräten verbunden sind.

\includegraphics[width=1.0\textwidth]{belege/grosstests/Bilder/Grosstest2/Test2Misserfolg2.jpg}

2. G3n3siS verschickt eine Nachricht, die jedoch von seinem Nachbarn
Nightfire nicht weitergeleitet wird.

\includegraphics[width=1.0\textwidth]{belege/grosstests/Bilder/Grosstest2/Test2Misserfolg3.jpg}

3. Nightfire baut eine Verbindung zu mehreren Nachbargeräten auf, die sich
jedoch nur gegenseitig finden.

\paragraph{Schlussfolgerung aus dem Test:}

Die Anzahl der erfolgreichen Verbindungen ist sehr stark von der
funktionierenden Verbindung zwischen mnex und Sven beeinflusst.

Die Tatsache, dass extrem viele Geräte überhaupt keine Verbindung mehr
aufbauen konnten, unterstützt die Aussage, die bereits nach Test 1
getroffen wurde, dass die Limitierung der Slots nicht zielführend war.


\subsubsection{Ergebnisse Test 3}

Da wir gemerkt haben, dass die Limitierung der Slots auf 3 nicht
zielführend war, haben wir diese für Test 3 wieder rückgängig gemacht.

Wir haben auf 164 Nachrichten 11 erfolgreiche Fälle und 153 Errors.

\paragraph{Folgende Beispiele veranschaulichen Nachrichten, die beim Empfänger angekommen sind:}

\includegraphics[width=1.0\textwidth]{belege/grosstests/Bilder/Grosstest2/Test3Erfolg1.jpg}

\includegraphics[width=1.0\textwidth]{belege/grosstests/Bilder/Grosstest2/Test3Erfolg2.jpg}

1 und 2. Nachrichten die zu einem benachbarten Empfänger gesendet werden.

\includegraphics[width=1.0\textwidth]{belege/grosstests/Bilder/Grosstest2/Test3Erfolg3.jpg}

3. Mnex sendet eine Nachricht an Sven diese kommt jedoch erst nach dem
zweiten Versuch an. Auch das Acknowledgement benötigt zwei Versuche.

\paragraph{Folgende Beispiele veranschaulichen Nachrichten, die nicht beim Empfänger angekommen sind:}

\includegraphics[width=1.0\textwidth]{belege/grosstests/Bilder/Grosstest2/Test3Misserfolg1.jpg}

1. G3n3siS sendet eine Nachricht. Diese geht jedoch nach zwei Hops verloren.\\

\includegraphics[width=1.0\textwidth]{belege/grosstests/Bilder/Grosstest2/Test3Misserfolg2.jpg}

2. Das wohl häufigste Beispiel des dritten Tests. Zwei benachbarte Geräte
schaffen es, eine Verbindung aufzubauen. Das zweite Handy schafft es
jedoch dann nicht, die Nachricht weiterzuleiten.

\includegraphics[width=1.0\textwidth]{belege/grosstests/Bilder/Grosstest2/Test3Misserfolg3.jpg}

3. Auch hier gab es Fälle, in denen ein Handy keine Verbindung mit einem anderen Gerät aufbauen konnte. Dies passiert jedoch deutlich seltener als bei Test 1 und 2.

\paragraph{Schlussfolgerung aus dem Test:}

Der dritte Test ist besser gelaufen als der 2. Da die Entfernung einen
erschwerenden Faktor darstellt, kann davon ausgegangen werden, dass die
Verbesserung dadurch zustande kam, dass wir die Slots nicht mehr
limitieren.

\subsubsection{Ausblick und zukünftige Tests}

Im Auftraggebertreffen wurde entschieden, dass keine weiteren
Verbesserungen am Bluetooth-Code durchgeführt werden sollen. Die Begründung
hierfür ist, dass die Implementierung des Bluetooth Stacks wohl
teilweise fehlerhaft ist.

Die Lösung hierfür ist eine eigene Implementierung des Bluetooth Stacks.
Diese soll jedoch im Rahmen einer Bachelorarbeit geschrieben werden, da
dies nicht Bestandteil des Bachelorpraktikums ist.

Da sich deshalb in der Funktionalität der Nachrichtenübertragung nichts
mehr ändern wird, haben wir auf einen dritten Großtest verzichtet.



\clearpage
\section{Erweiterbarkeit}

Während der Entwicklung der Anwendung haben wir zusammen mit unserem Auftraggeber als weiteres Qualitätsziel erkannt, dass die App leicht um weitere Übertragungswege erweiterbar sein soll. Zu diesem Zweck existiert ein Interface `IProtocol`.

Die Erweiterung der Anwendung um einen weiteren Übertragungsweg wird im folgenden anhand des beispielhaften EchoProtocol dokumentiert, welches Nachrichten direkt an den Absender zurückschickt. Dazu sind folgende Schritte notwendig:

\subsection{Entwicklung einer neuen Klasse `EchoProtocol`}

Der Entwickler muss zunächst eine neue Klasse EchoProtocol implementieren. Die neue Klasse muss das Interface `IProtocol` implementieren.

\begin{lstlisting}[language=Java]
app/src/main/java/de/tudarmstadt/informatik/bp/bonfirechat/network/IProcol.java

public interface IProtocol {
    void sendPacket(Packet packet, Peer peer);
    boolean canSend();
    void shutdown();
}
\end{lstlisting}

Eine beispielhafte Implementierung des EchoProtocol sieht wie folgt aus:

\begin{lstlisting}[language=Java]
app/src/main/java/de/tudarmstadt/informatik/bp/bonfirechat/network/EchoProtocol.java

public class EchoProtocol extends SocketProtocol {

    private static final String TAG = "EchoProtocol";

    private Identity mIdentity;
    private OnPacketReceivedListener mOnPacketReceivedListener;
    private OnPeerDiscoveredListener mOnPeerDiscoveredListener;

    public EchoProtocol(Context ctx) {
    }

    // ###########################################################################
    // ###    Implementation of IProtocol
    // ###########################################################################

    void setIdentity(Identity identity) {
         mIdentity = identity;
    }
    void setOnPacketReceivedListener(OnPacketReceivedListener listener) {
         mOnPacketReceivedListener = listener;
    }
    void setOnPeerDiscoveredListener(OnPeerDiscoveredListener listener) {
         mOnPeerDiscoveredListener = listener;
    }
    
    @Override
    public void sendPacket(Packet packet, Peer peer) {
        Log.d(TAG, "echoing packet via EchoProtocol");
        mOnPacketReceivedListener.onPacketReceived(this, packet);
    }

    @Override
    public boolean canSend() {
        // this dummy protocol is always ready to send messages
        return true;
    }

    @Override
    public void shutdown() {
    }
}

\end{lstlisting}


\subsection{Eintragen des neuen Protokolls im ConnectionManager}

Damit das Protokoll beim Start der Anwendung instanziiert werden kann, muss es in der Liste der registrierten Protokolle im ConnectionManager aufgeführt werden:
\begin{lstlisting}[language=Java]
app/src/main/java/de/tudarmstadt/informatik/bp/bonfirechat/network/ConnectionManager.java

public class ConnectionManager extends NonStopIntentService {

    // ...
    
    static final Class[] registeredProtocols = new Class[]{
            BluetoothProtocol.class,
            WifiProtocol.class,
            GcmProtocol.class,
            EchoProtocol.class   // new
            };

\end{lstlisting}

Außerdem sollte in den App-Einstellungen ein neues Auswahlfeld eingefügt werden, sodass der Benutzer das Protokoll ein- und ausschalten kann.

\begin{lstlisting}
app/src/main/res/xml/preferences.xml

        <CheckBoxPreference android:key="enable_EchoProtocol" android:defaultValue="true"
            android:title="@string/use_echo_protocol"></CheckBoxPreference>
\end{lstlisting}




\subsection{Quellcode BluetoothProtocol}

Als Beispiel für eine vollständige Implementierung eines Übertragungsprotokolls führen wir hier noch den Quellcode des BluetoothProtocol sowie der Elternklasse SocketProtocol, die einige Methoden enthält, die von allen bisherigen Protokoll-Implementierungen verwendet werden, und ihrerseits von IProtocol erbt, an.

\subsubsection{SocketProtocol}
\lstinputlisting[language=Java]{../BonfireChat/app/src/main/java/de/tudarmstadt/informatik/bp/bonfirechat/network/SocketProtocol.java}

\subsubsection{BluetoothProtocol}
\lstinputlisting[language=Java]{../BonfireChat/app/src/main/java/de/tudarmstadt/informatik/bp/bonfirechat/network/BluetoothProtocol.java}




