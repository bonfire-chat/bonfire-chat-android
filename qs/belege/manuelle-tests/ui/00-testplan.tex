
Die einzelnen Testabläufe sind nach funktionalen Bereichen der App
aufgeteilt.


% ------------------------------------------------------------------------
% I

\subsubsection{UI-i: Darstellung}\label{i-darstellung}

\paragraph{UI-i-1. Darstellung der
Kontaktübersicht}\label{darstellung-der-kontaktuxfcbersicht}

Die Kontaktübersicht(ContactsActivity) soll so übersichtlich wie möglich
gehalten sein. Dazu sind einige Dinge zu beachten.

\begin{longtable}{|p{8cm}|p{8.5cm}|}
\hline
Benutzeraktion & Erwartetes Verhalten der App\tabularnewline
\hline

Der Benutzer öffnet die Kontaktübersicht & Dem Benutzer werden alle
Kontakte aufsteigend alphabetisch angezeigt, also oben beginnend mit
Kontaktnamen die mit A anfangen usw.\tabularnewline
\hline
\end{longtable}

\paragraph{UI-i-2. Darstellung der
Unterhaltungsübersicht}\label{darstellung-der-unterhaltungsuxfcbersicht}

Die Unterhaltungsübersicht(ConversationsActivity) soll so übersichtlich
wie möglich gehalten sein. Dazu sind einige Dinge zu beachten.

\begin{longtable}{|p{8cm}|p{8.5cm}|}
\hline
Benutzeraktion & Erwartetes Verhalten der App\tabularnewline
\hline

Der Benutzer öffnet die Unterhaltungsübersicht & Dem Benutzer werden
alle Unterhaltungen angezeigt, und zwar geordnet nach der Zeit der
letzten Nachricht, beginnend mit der Unterhaltung mit der neuesten
Nachricht. Die aktuellste Nachricht wird unter dem Namen der
Unterhaltung angezeigt\tabularnewline
\hline
\end{longtable}

\paragraph{UI-i-3. Darstellung der
Chatansicht}\label{darstellung-der-chatansicht}

Der Benutzer soll möglicht übersichtlich Nachrichten lesen können.
Dementsprechend muss die Chatansicht (MessagesActivity) gestaltet sein.

\paragraph{UI-i-4. Allgemeine Darstellung der
Nachrichten}\label{allgemeine-darstellung-der-nachrichten}

Nachrichten in der Chatansicht werden grundsätzlich wie hier beschrieben
dargestellt. Die neuesten Nachrichten sind unten zu finden, die ältesten
oben. Nachrichten des Benutzers werden rechts angezeigt, die seines
Chatpartners links. Neben der Nachricht wird das Kontaktbild des
Absenders angezeigt. Unter dem Nachrichtentext sind durch folgende
Symbole die Nachrichtendetail dargestellt:

\begin{itemize}
\tightlist
\item
  Ein Schlosssymbol, wenn die Nachricht verschlüsselt ist
\item
  Das Bluetooth-Symbol, wenn die Nachricht über Bluetooth versendet
  wurde
\item
  Das WiFi-Symbol, wenn die Nachricht über WiFi peer-to-peer versendet
  wurde
\item
  Eine Wolke, wenn die Nachricht über Google-Cloud-Messaging versendet
  wurde
\item
  Ein rotes Dreieck mit Ausrufezeichen darin, wenn die Nachricht nicht
  angekommen ist
\end{itemize}

Außerdem wird noch der genaue Sendezeitpunkt angezeigt.

\begin{longtable}{|p{8cm}|p{8.5cm}|}
\hline
Benutzeraktion & Erwartetes Verhalten der App\tabularnewline
\hline

Der Benutzer öffnet eine Unterhaltung & Die Tastatur öffnet sich und der
Benutzer kann einen Text eingeben. Gleichzeitig werden die vergangenen
Nachrichten des Chats angezeigt wie in
\href{Manueller-Test-UI\#allgemeine-darstellung-der-nachrichten}{Allgemeine
Darstellung von Nachrichten} beschrieben\tabularnewline
\hline
\end{longtable}


% ------------------------------------------------------------------------
% II

\clearpage
\subsubsection{UI-ii: Erster Start und
Einstellungen}\label{ii-erster-start-und-einstellungen}

\paragraph{UI-ii-1. Willkommensseite}\label{willkommensseite}

\begin{longtable}{|p{8cm}|p{8.5cm}|}
\hline
~Benutzeraktion & Erwartetes Verhalten der App\tabularnewline
\hline

Die App wird nach Neuinstallation / Löschen der App-Daten geöffnet. & Im
Hintergrund wird ein Schlüsselpaar für das Gerät generiert. Es öffnet
sich eine Begrüßungsseite, auf der der Nutzer zur Eingabe eines Namens
(Nickname) aufgefordert wird.\tabularnewline
Es wird ein Nickname eingetragen, der bereits vergeben ist. Der
Bestätigungsbutton wird betätigt. & Es wird ein HTTP-Post-Request an das
Server-API gesendet, welcher Public-Key, Nickname und optional
Handynummer enthält.\tabularnewline
~ & Es wird eine Fehlermeldung angezeigt, die den Nutzer auffordert,
einen anderen Nickname zu wählen. Die Begrüßungsseite bleibt
sichtbar.\tabularnewline
Es wird ein neuer Nickname eingetragen. Der Bestätigungsbutton wird
betätigt. & Die Begrüßungsseite schließt sich und der Benutzer gelangt
in die Haupt-Activity der App, wobei der Navigations-Drawer geöffnet
ist.\tabularnewline
Die App wird durch Betätigen der Back-Taste verlassen. &\tabularnewline
\hline
\end{longtable}

\paragraph{UI-ii-2. Automatisches Öffnen des
NavigationDrawer}\label{automatisches-uxf6ffnen-des-navigationdrawer}

Um dem Benutzer die Navigation in der App nahezubringen, soll, wie in
den Android-UI-Richtlinien empfohlen, das Navigationsmenü
(NavigationDrawer) so lange beim App-Start automatisch eingeblendet
werden, bis der Benutzer es einmal selbstständig geöffnet hat.

\begin{longtable}{|p{8cm}|p{8.5cm}|}
\hline
Benutzeraktion & Erwartetes Verhalten der App\tabularnewline
\hline

~Die App wird aus der App-Liste gestartet & Die MainActivity der App
öffnet sich, der Menüpunkt ``Unterhaltungen'' ist ausgewählt, der
NavigationDrawer ist geöffnet.\tabularnewline
Schließen des NavigationDrawer durch Klick auf ``Unterhaltungen''.
Wieder Öffnen des NavigationDrawer durch Hereinwischen vom linken
Bildschirmrand. & Der NavigationDrawer schließt sich und öffnet sich
wieder.\tabularnewline
Beenden der App durch Betätigen der Back-Taste &\tabularnewline
Die App wird erneut aus der App-Liste gestartet & Die MainActivity der
App öffnet sich, der Menüpunkt ``Unterhaltungen'' ist sichtbar, der
NavigationDrawer ist \textbf{nicht} geöffnet.\tabularnewline
Beenden der App durch Betätigen der Back-Taste &\tabularnewline
\hline
\end{longtable}

\paragraph{UI-ii-3. Tutorial beim ersten
Start}\label{tutorial-beim-ersten-start}

Um dem Benutzer den Einstieg in die Benutzung der App zu vereinfachen,
wird er beim ersten Start der App durch ein interaktives Tutorial
geführt. Dieses muss einwandfrei funktionieren, damit der Nutzer nicht
verwirrt wird.

\begin{longtable}{|p{8cm}|p{8.5cm}|}
\hline
Benutzeraktion & Erwartetes Verhalten der App\tabularnewline
\hline

Die App wird nach Neuinstallation / Löschen der App-Daten geöffnet. Der
Benutzer gibt einen noch freien Benutzernamen und seine Telefonnummer
ein und drückt auf OK & Dem Benutzer wird der NavigationDrawer gezeigt
und darauf hingewiesen, dass er mit diesem in der App navigieren kann.
Des Weiteren wird ein Knopf eingeblendet, mit dem der Benutzer zum
nächsten Teil des Tutorials springen kann\tabularnewline
Der Benutzer drückt auf den eben beschriebenen Knopf & Es wird der
Share-Button zum Teilen der Kontaktinformationen hervorgehoben, und dem
Benutzer seine Funktionsweise beschrieben. Des Weiteren wird ein Knopf
eingeblendet, mit dem der Benutzer zum nächsten Teil des Tutorials
springen kann\tabularnewline
Der Benutzer drückt auf den eben beschriebenen Knopf & Der Benutzer wird
aufgefordert, auf Unterhaltungen zu klicken\tabularnewline
Der Benutzer klickt auf Unterhaltungen & Die Unterhaltungsübersicht
öffnet sich, und es ist bereits eine Unterhaltung zu sehen. Der Benutzer
wird aufgefordert, die Unterhaltung anzuklicken\tabularnewline
Der Benutzer klickt die Unterhaltung an & Die Unterhaltung öffnet sich
und der Benutzer wird als erstes auf die Titel-ändern-Funktion
hingewiesen. Auch wird ihm wieder ein Weiter-Knopf
angezeigt\tabularnewline
Der Benutzer drückt auf den eben beschriebenen Knopf & Der Benutzer wird
nun auf die Bilder-schicken-Funktion hingewiesen. Auch wird ihm wieder
ein Weiter-Knopf angezeigt\tabularnewline
Der Benutzer drückt auf den eben beschriebenen Knopf & Der Benutzer wird
auf die Standort-Teilen-Funkiton hingewiesen. Auch wird ihm wieder ein
Weiter-Knopf angezeigt\tabularnewline
Der Benutzer drückt auf den eben beschriebenen Knopf & Eine Nachricht
wird hervorgehoben und dem Benutzer erklärt, dass er durch Tippen auf
die Nachricht in die Nachrichtendetailansicht gelangen kann. Auch wird
wieder ein Weiter-Knopf angezeigt\tabularnewline
Der Benutzer drückt auf den eben beschriebenen Knopf & Der
NavigationDrawer wird wieder eigeblendet und der Benutzer wird
aufgefordert, auf Kontakte zu klicken.\tabularnewline
Der Benutzer klickt auf Kontakte & Die Kontaktübersicht öffnet sich und
der Benutzer wird auf die Kontaktdaten-teilen-Funktion hingewiesen. Auch
wird wieder ein Weiter-Knopf angezeigt.\tabularnewline
Der Benutzer drückt auf den eben beschriebenen Knopf & Der Benutzer wird
auf die QR-Code-Scannen-Funktion hingewiesen. Auch wird wieder ein
Weiter-Knopf angezeigt.\tabularnewline
Der Benutzer drückt auf den eben beschriebenen Knopf & Der Benutzer wird
auf die Kontakte-Suchen-Funktion hingewiesen. Nun wird ein ``Los
gehts''-Knopf angezeigt.\tabularnewline
Der Benutzer drückt auf den eben beschriebenen Knopf & Der Benutzer
gelangt wieder in die Unterhaltungsübersicht. Das Tutorial ist
beendet.\tabularnewline
\hline
\end{longtable}


% ------------------------------------------------------------------------
% III

\clearpage
\subsubsection{UI-iii: Senden und Empfangen von
Nachrichten}\label{iii-senden-und-empfangen-von-nachrichten}

\paragraph{UI-iii-1. Korrektes Anzeigen von
Benachrichtigungen}\label{korrektes-anzeigen-von-benachrichtigungen}

Benachrichtigungen über neue Nachrichten sollen nur angezeigt werden,
wenn das entsprechende Chatfenster nicht geöffnet ist. Außerdem sollen
diese Benachrichtigungen automatisch verschwinden, wenn die betreffende
Nachricht gelesen wurde.

\begin{longtable}{|p{8cm}|p{8.5cm}|}
\hline
Benutzeraktion & Erwartetes Verhalten der App\tabularnewline
\hline

Der Benutzer befindet sich auf dem Startbildschirm der App, eine
Nachricht trifft ein & Es wird eine Benachrichtigung in der
Benachrichtigungsleiste angezeigt, beim Klicken auf die Benachrichtigung
gelangt der Benutzer zum Chat, in dem die Nachricht gesendet wurde, die
Benachtrichtigung verschwindet\tabularnewline
Der Benutzer befindet sich in einem Chat (in der MessagesActivity), eine
Nachricht trifft in diesem Chat ein. & Es wird keine Benachrichtigung
angezeigt\tabularnewline
Der Benutzer befindet sich in einem Zustand, in dem eine
Benachrichtigung über eine Nachricht angezeigt wird. Dann liest der die
entsprechende Nachricht, indem er direkt - ohne anklicken der
Benachrichtigung - in den entsprchenden Chat geht & Die Benachrichtigung
verschwindet\tabularnewline
\hline
\end{longtable}

\paragraph{UI-iii-2. Senden und empfangen von normalen
Nachrichten}\label{senden-und-empfangen-von-normalen-nachrichten}

\begin{longtable}{|p{8cm}|p{8.5cm}|}
\hline
Benutzeraktion & Erwartetes Verhalten der App\tabularnewline
\hline

Benutzer befindet sich in der Chatansicht und tippt einen
Nachrichtentext in das Textfeld ein. Dann drückt er die Sendetaste & Die
Nachricht wird sofort an den Chatpartner gesendet. Die Nachricht wird
sofort angezeigt, wie in
\href{Manueller-Test-UI\#allgemeine-darstellung-der-nachrichten}{Allgemeine
Darstellung von Nachrichten} beschrieben.\tabularnewline
Der Benutzer empfängt eine Nachricht und geht dann in die entsprechende
Chatansicht oder ist dort bereits & Alle Nachrichten werden wie in
\href{Manueller-Test-UI\#allgemeine-darstellung-der-nachrichten}{Allgemeine
Darstellung von Nachrichten} angezeigt. Die Empfangene Nachricht ist
bereits enthalten.\tabularnewline
\hline
\end{longtable}

\paragraph{UI-iii-3. Senden und empfangen von
Bildern}\label{senden-und-empfangen-von-bildern}

Das Senden von Bildern soll mögichst einfach in der Bedienung sein und
den Benutzer nicht verwirren. Vor allem soll die App nicht bei
unerwarteten Eingaben abstürzen.

\begin{longtable}{|p{8cm}|p{8.5cm}|}
\hline
Benutzeraktion & Erwartetes Verhalten der App\tabularnewline
\hline

Der Benutzer klickt in der Chatansicht auf das Bild-Senden-Symbol & Es
öffnet sich eine von Android bereitgestellte Liste mit versendbaren
Bildern.\tabularnewline
Der Benutzer klickt auf ein Bild in der eben beschriebenen Liste & Das
Bild wird sofort an den Chatpartner gesendet. Die App kehrt zu
betreffenden Chat zurück. Dort ist nun das Bild zu sehen.\tabularnewline
Der Benztzer wählt in der Liste kein Bild aus und kehrt mittels des
Zurück-Knopfs zum Chat zurück & Es passiert nichts, insbesondere stürzt
die App nicht ab.\tabularnewline
Der Benutzer versucht, ein nicht darstellbares Bild zu versenden & Es
wird eine Fehlermeldung angezeigt, dass das Bild nicht versendet werden
kann. Das Bild wird nicht versendet. Die App kehrt zur MessagesActivity
zurück.\tabularnewline
Der Benutzer bekommt in einem Chat ein Bild gesendet. Daraufhin wechselt
er in diesen Chat & Der Benutzer sieht das ihm gesendete
Bild\tabularnewline
\hline
\end{longtable}

\paragraph{UI-iii-4. Teilen des aktuellen Standorts}\label{teilen-des-aktuellen-standorts}

\begin{longtable}{|p{8cm}|p{8.5cm}|}
\hline
Benutzeraktion & Erwartetes Verhalten der App\tabularnewline
\hline

Der Benutzer befindet sich in der Chatansicht, er tippt auf das
Kompass-Symbol in der Menüleiste. Sein Standort kann bestimmt werden &
Der Standort wird zum Chatpartner gesendet. Es ist eine
Google-Maps-Vorschau des Standorts zu sehen\tabularnewline
Der Benutzer befindet sich in der Chatansicht, er tippt auf das
Kompass-Symbol in der Menüleiste. Sein Standort kann nicht bestimmt
werden. & Es wird eine Fehlermeldung eingeblendet, die aussagt, dass der
Standort nicht bestimmt werden kann und deswegen auch nicht gesendet
werden kann\tabularnewline
Der Benutzer empfängt eine Standortteilung & Die App verhält sich wie
bei normalen Nachrichten. In der Chatansicht sieht er eine
Google-Maps-Vorschau des Standorts\tabularnewline
Der Benutzer tippt auf eine Vorschau eines Standorts & Eine neue Ansicht
öffnet sich, in der der Standort auf einer Karte in Großansicht
angezeigt wird.\tabularnewline
\hline
\end{longtable}



% ------------------------------------------------------------------------
% IV

\clearpage
\subsubsection{UI-iv: Löschen von Elementen}\label{iv-luxf6schen-von-elementen}

\paragraph{UI-iv-1. Löschen von Kontakten}\label{luxf6schen-von-kontakten}

\begin{longtable}{|p{8cm}|p{8.5cm}|}
\hline
Benutzeraktion & Erwartetes Verhalten der App\tabularnewline
\hline

Der Benutzer ist in der Kontaktansicht und wählt durch langes Tippen auf
einen Kontakt denselben aus & Die Menüleiste wird durch eine
auswahlspezifische Menüleiste ersetzt, in der auch ein Mülleimer als
Lösch-Symbol zu sehen ist\tabularnewline
Der Benutzer wählt einen Kontakt aus und löscht ihn durch Tippen auf das
Mülleimer-Symbol & Der Kontakt wird aus der Datenbank gelöscht und
verschwindet sofort aus der Ansicht. Mit dem Kontakt werden auch
sämtliche mit dem Kontakt geführte Unterhaltungen gelöscht, die ab
sofort nicht mehr in der Unterhaltungsübersicht zu sehen
sind\tabularnewline
Nach dem Löschen eines spezifischen Kontakts wechselt der Benutzer auf
eine andere Ansicht und öffnet dann die Kontaktansicht erneut. & Der
vorher gelöschte Kontakt ist immer noch nicht zu sehen\tabularnewline
\hline
\end{longtable}

\paragraph{UI-iv-2. Löschen von
Unterhaltungen}\label{luxf6schen-von-unterhaltungen}

\begin{longtable}{|p{8cm}|p{8.5cm}|}
\hline
Benutzeraktion & Erwartetes Verhalten der App\tabularnewline
\hline

Der Benutzer ist in der Unterhaltungsansicht und wählt durch langes
Tippen auf eine Unterhaltung dieselbe aus & Die Menüleiste wird durch
eine auswahlspezifische Menüleiste ersetzt, in der auch ein Mülleimer
als Lösch-Symbol zu sehen ist\tabularnewline
Der Benutzer wählt eine Unterhaltung aus und löscht sie durch Tippen auf
das Mülleimer-Symbol & Die Unterhaltung wird aus der Datenbank gelöscht
und verschwindet sofort aus der Ansicht.\tabularnewline
Nach dem Löschen einer spezifischen Unterhaltung wechselt der Benutzer
auf eine andere Ansicht und öffnet dann die Unterhaltungsansicht erneut.
& Die vorher gelöschte Unterhaltung ist immer noch nicht zu
sehen\tabularnewline
\hline
\end{longtable}

\paragraph{UI-iv-3. Löschen von
Nachrichten}\label{luxf6schen-von-nachrichten}

\begin{longtable}{|p{8cm}|p{8.5cm}|}
\hline
Benutzeraktion & Erwartetes Verhalten der App\tabularnewline
\hline

Der Benutzer ist in der Chatansicht und wählt durch langes Tippen auf
eine Nachricht dieselbe aus & Die Menüleiste wird durch eine
auswahlspezifische Menüleiste ersetzt, in der auch ein Mülleimer als
Lösch-Symbol zu sehen ist\tabularnewline
Der Benutzer wählt eine Nachricht aus und löscht sie durch Tippen auf
das Mülleimer-Symbol & Die Nachricht wird aus der Datenbank gelöscht und
verschwindet sofort aus der Ansicht.\tabularnewline
Nach dem Löschen einer spezifischen Nachricht wechselt der Benutzer auf
eine andere Ansicht und öffnet dann die Chatansicht erneut. & Die vorher
gelöschte Nachricht ist immer noch nicht zu sehen\tabularnewline
\hline
\end{longtable}


% ------------------------------------------------------------------------
% V

\clearpage
\subsubsection{UI-v: Diverses}\label{v-diverses}

\paragraph{UI-v-1. Korrekte Funktionsweise der regelmäßigen
Standortteilung}\label{korrekte-funktionsweise-der-regelmuxe4uxdfigen-standortteilung}

Die Ansicht, unter der man den dauerhaft geteilten Standort eines
Chatpartners sehen kann, soll automatisch aktualisiert werden, wenn ein
Standortupdate eintrifft.

\begin{longtable}{|p{8cm}|p{8.5cm}|}
\hline
Benutzeraktion & Erwartetes Verhalten der App\tabularnewline
\hline

Der Benutzer ist in der Kontaktansicht und wählt durch langes Tippen auf
einen Kontakt denselben aus & Die Menüleiste wird durch eine
auswahlspezifische Menüleiste ersetzt, in der auch ein Standort-Symbol
zu sehen ist\tabularnewline
Der Benutzer b1 hat den Kontakt b2 ausgewählt und tippt auf das
Standort-Symbol. Der b2 hat seinen Standort noch nicht geteilt & Es wird
dem Benutzer b1 ein Ansicht angezeigt, die ihm sagt, dass kein Standort
vorliegt\tabularnewline
Benutzer b1 befindet sich in der eben beschriebenen Ansicht. b2 geht nun
in das Kontakt-bearbeiten-Menü von b1 und setzt das ``Diesem Kontakt
meinen Standort mitteilen''-Häkchen & Die von b1 geöffnete Ansicht zeigt
nun eine Karte, auf der der Standort von b2 zu sehen ist.\tabularnewline
b1 befindet sich in der Ansicht, in der der Standort von b2 zu sehen
ist. b2 entfernt sich um mindestens 50m von seinem Ausganspunkt. & b1
sieht nun innerhalb des aktuellen Aktualisierungsintervalls des
Standorts eine veränderte Position von b2 auf der Karte\tabularnewline
\hline
\end{longtable}
