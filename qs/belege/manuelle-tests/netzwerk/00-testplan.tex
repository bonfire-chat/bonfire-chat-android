
Dieser Abschnitt beschreibt einen Testplan für manuelles Testen der
drahtlosen Übertragung sowie des Routing-Algorithmus.\\\\

% ----------------------------------------------------------------------

\subsubsection{N-i Definitionen}

\paragraph{Allgemeine Ausgangskonfiguration der Knoten}

Ein Knoten ist, soweit nicht näher beschrieben, ein Android-Gerät, auf dem die aktuellste Version der App eingerichtet und die initiale Einrichtung (Eingabe eines Nicknames, dadurch Registrierung des Gerätes mit Nickname, GCM-ID und PublicKey am Server) abgeschlossen ist.

% ----------------------------------------------------------------------
% II


\subsubsection{N-ii Testfälle für die allgemeine Übertragung}

\paragraph{N-ii-1 Test der direkten Übertragung via Flooding / Bluetooth}

\begin{longtable}{p{8cm}p{8.5cm}}
\toprule
Benutzerinteraktion & erwartetes Verhalten der App\tabularnewline
\midrule
\endhead
Auf zwei Knoten A und B, die sich in direkter Bluetooth-Reichweite
befinden, wird zunächst die Ausgangskonfiguration hergestellt,
anschließend werden in den Einstellungen alle Übertragungsverfahren
außer Bluetooth deaktiviert. Die Kontaktdaten von A werden an B
gesendet. Auf B wird eine neue Unterhaltung mit A gestartet. Auf B wird
eine Nachricht an A eingegeben und abgesendet. & B sendet die Nachricht
an alle per Bluetooth sichtbaren Geräte, insbesondere an A. Auf A wird
die Nachricht empfangen und als per Bluetooth empfangen dargestellt. A
sendet ein ACK an B, dieses wird auf B durch ein Häkchen an der
Nachricht sichtbar. Weiterhin erscheint ein Bluetooth-Icon an der
Nachricht, da die Nachricht per Bluetooth zugestellt wurde. Im Dashboard
ist erkennbar, dass die Nachricht mit Routingmodus Flooding (0x01)
versendet wurde, das ACK mit Routingmodus DSR (0x02).\tabularnewline
\bottomrule
\end{longtable}

\paragraph{N-ii-2 Test der direkten Übertragung via DSR / Bluetooth}

\begin{longtable}{p{8cm}p{8.5cm}}
\toprule
Benutzerinteraktion & erwartetes Verhalten der App\tabularnewline
\midrule
\endhead
Nach der erfolgreichen Durchführung von Test 1 wird eine weitere
Nachricht von B an A gesendet, sowie eine Nachricht von A an B. & Nach
Test 1 ist den Geräten A und B der schnellste Pfad zum jeweils anderen
Gerät bekannt. Daher sendet B die Nachricht nicht per Routingmodus
Flooding (0x01), sondern per Dynamic Source Routing (0x02), also unter
der Angabe der gewünschten Übertragungspfades. Daher wird die Nachricht
nur an Gerät A gesendet. Darüber hinaus identisch zu Test
1.\tabularnewline
\bottomrule
\end{longtable}

\paragraph{N-ii-3 Test der direkten Übertragung via Server / Google Cloud Messaging}

\begin{longtable}{p{8cm}p{8.5cm}}
\toprule
Benutzerinteraktion & erwartetes Verhalten der App\tabularnewline
\midrule
\endhead
Auf zwei Knoten A und B, die eine funktionierende Internetverbindung
haben, wird zunächst die Ausgangskonfiguration hergestellt, anschließend
werden in den Einstellungen alle Übertragungsverfahren außer \glqq mobile
Daten / Wifi\grqq~ (Übertragung per Cloud) deaktiviert. Die Kontaktdaten von
A werden an B gesendet. Auf B wird eine neue Unterhaltung mit A
gestartet. Auf B wird eine Nachricht an A eingegeben und abgesendet. & B
sendet die Nachricht per HTTP an den Server, welcher sie per GCM an
Gerät A weiterleitet. Auf A wird die Nachricht empfangen und als per
Cloud empfangen dargestellt. A sendet ein ACK an B, dieses wird auf B
durch ein Häkchen an der Nachricht sichtbar. Weiterhin erscheint ein
Cloud-Icon an der Nachricht, da die Nachricht per Cloud zugestellt
wurde. Im Dashboard ist erkennbar, dass die Nachricht mit Routingmodus
Flooding (0x01) versendet wurde, das ACK mit Routingmodus DSR
(0x02).\tabularnewline
\bottomrule
\end{longtable}

% ----------------------------------------------------------------------
% III


\subsubsection{N-iii Testfälle für Wegfindung per Flooding und einfaches Routing}

In diesen Tests werden folgende Teile des Routing getestet:

\begin{itemize}
  \item Finden des schnellsten Pfades: Flooding an alle Knoten sowie ACK auf dem Pfad, auf dem die Nachricht den Empfänger zuerst erreicht (schnellster Pfad)
  \item Speichern des schnellsten Pfades zu einem Empfänger
  \item Verwenden des gespeicherten schnellsten Pfades für künftige Nachrichten
\end{itemize}

\paragraph{N-iii-1 Bluetooth-Wegfindung - Flooding mit drei Knoten}

\begin{longtable}{p{8cm}p{8.5cm}}
\toprule
Benutzerinteraktion & erwartetes Verhalten der App\tabularnewline
\midrule
\endhead
Auf drei Knoten A, B und C werden alle Übertragungsverfahren außer
Bluetooth deaktiviert. Die Kontaktdetails von C werden an A gesendet.
Die Knoten werden räumlich so angeordnet, dass eine Bluetoothverbindung
zwischen A und B sowie zwischen B und C möglich ist, nicht jedoch
zwischen A und C. Auf A wird eine Unterhaltung mit C begonnen und eine
Nachricht an C gesendet. & Nachricht u. ACK kommen an,
etc.\tabularnewline
\bottomrule
\end{longtable}

\paragraph{N-iii-2 Bluetooth-Wegfindung zum nächsten Knoten mit Internetverbindung}

\begin{longtable}{p{8cm}p{8.5cm}}
\toprule
Benutzerinteraktion & erwartetes Verhalten der App\tabularnewline
\midrule
\endhead
Auf den Knoten A und B werden alle Übertragungsverfahren außer Bluetooth deaktiviert. 
Auf B wird GCM zusätzlich aktiviert. Auf Knoten C wird nur GCM aktiviert. 
Die Kontaktdetails von C werden an A gesendet. 
Die Knoten werden räumlich so angeordnet, dass eine Bluetoothverbindung zwischen A und B möglich ist. 
Auf A wird eine Unterhaltung mit C begonnen und eine Nachricht an C gesendet.
& A sendet die Nachricht per Bluetooth an B.
B sendet die Nachricht per GCM an C. 
Das ACK kommt auf umgekehrtem Weg zu A zurück.\tabularnewline
\bottomrule
\end{longtable}

%\paragraph{N-iii-3.}\label{section}

%TODO Test beschreiben

\clearpage


% ------------------------------------------------------------------------
% IV

\subsubsection{N-iv Testfälle für sich verändernde Routen}

In diesen Tests wird überprüft, ob der Routingalgorithmus bei ausfallenden Pfaden korrekt reagiert.

\paragraph{N-iv-1 Abreißende Bluetooth-Verbindung}

\begin{longtable}{p{8cm}p{8.5cm}}
\toprule
Benutzerinteraktion & erwartetes Verhalten der App\tabularnewline
\midrule
\endhead
Auf Knoten A wird nur Bluetooth aktiviert, auf B werden GCM und
Bluetooth aktiviert. Von A wird eine Nachricht \glqq alpha\grqq~ an B gesendet.
Nachdem diese zugestellt und durch Ack bestätigt wurde, wird eine weitere
Nachricht \glqq beta\grqq~ von A an B gesendet. Danach wird A räumlich so weit
vom B entfernt, dass keine Bluetooth-Übertragung mehr möglich ist.
Weiterhin wird auf A die Übertragung per GCM aktiviert. Anschließend
wird eine weitere Nachricht \glqq gamma\grqq~ von A and B gesendet. & Die
Nachricht \glqq alpha\grqq~ wird per Flooding gesendet, B sendet per DSR ein ACK
\glqq für alpha\grqq~ zurück. Danach hat A den Pfad zu B gespeichert und die
Nachricht \glqq beta\grqq~ wird direkt per DSR an B gesendet. Die Nachricht
\glqq gamma\grqq~ wird auch versucht per DSR direkt via Bluetooth an B zu
senden. Da dies fehlschlägt, wird beim Retry nach 20 Sekunden versucht,
\glqq gamma\grqq~ per Flooding zuzustellen. Dies erfolgt dann über GCM. Das ACK
\glqq für gamma\grqq~ von B an A erfolgt dann wiederum per DSR.\tabularnewline
\bottomrule
\end{longtable}
\clearpage


% ------------------------------------------------------------------------
% V

%\clearpage
%\subsubsection{N-v: Testfälle für Störeinflüsse im
%Netzwerk}\label{v-testfuxe4lle-fuxfcr-stuxf6reinfluxfcsse-im-netzwerk}

%Hier soll überprüft werden, wie sich die App bei äußeren Störeinflüssen
%auf Netzwerkebene, also zum Beispiel während einer während der
%Datenübertragung abreißenden Verbindung, verhält. Es soll sichergestellt
%sein, dass stets angemessen reagiert wird und sowohl die App nicht
%abstürzt als auch der Benutzer sinnvolle Fehlermeldungen erhält.

%Da dies teilweise sporadische Fehler sind, welche sich schwer
%reproduzieren lassen, ist die Aussagekraft der folgenden Tests leider
%nur bedingt gegeben.

%\paragraph{\texorpdfstring{N-v-1. ``Socket wird unerwartet während der
%Verbindung
%geschlossen''}{N-v-1. Socket wird unerwartet während der Verbindung geschlossen}}\label{socket-wird-unerwartet-wuxe4hrend-der-verbindung-geschlossen}

%\begin{longtable}{p{8cm}p{8.5cm}}
%\toprule
%Benutzerinteraktion & erwartetes Verhalten der App\tabularnewline
%\midrule
%\endhead
%\bottomrule
%\end{longtable}

%\paragraph{N-v-2. Benutzer schaltet Bluetooth im Telefon
%aus}\label{benutzer-schaltet-bluetooth-im-telefon-aus}

%\begin{longtable}{p{8cm}p{8.5cm}}
%\toprule
%Benutzerinteraktion & erwartetes Verhalten der App\tabularnewline
%\midrule
%\endhead
%\bottomrule
%\end{longtable}
