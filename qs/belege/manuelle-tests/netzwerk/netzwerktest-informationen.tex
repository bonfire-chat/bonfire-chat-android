\subsection{Teil N-i Allgemeine Informationen zu den Testergebnissen}

\paragraph{Test N-i.1 Packet Attribute im Dashboard}

Die Spaltenfunktionalit�t im Dashboard ist der Reihe nach aufgelistet:

\begin{itemize}
  \item Ger�t, das die Informations ans Dashboard sendet
  \item Ereignis, das den Informationsautsch ausl�st
  \item Uhrzeit des Ereignis
  \item Verwendetes �bertragungsprotokoll
  \item Adresse des Empf�ngers(Falls Ereignis = ``SEND'')
  \item Packetdetails
\end{itemize}

In der sechsten Spalte im Dashboard werden Details der �bertragung dargestellt. Dabei ist die eindeutige ID der Nachricht das zweite Attribut vom Packet. Desweiteren steht \glqq routing=1\grqq~ f�r Senden via Flooding und \glqq routing=2\grqq~ f�r Senden via DRS-Adaption.

\paragraph{Test N-i.2 Zeiten der Nachrichten im Dashboard}

Die angegebenen Zeiten der Ereignisse im Dashboard wirken auf den ersten Blick unwahrscheinlich. Zum Beispiel wird in Durchf�hrung des manuellen Netzwerk-Testplans vom 30. August 2015 Test II) 1 ein Acknowledgement zuerst empfangen, bevor es gesendet wurde. Dies liegt daran, dass die �bertragungn des Acknowledgements per Bluetooth stattgefunden hat und das Ger�t, das diese empfangen hat, eine schnellere Verbindung zum Dashboard-Server hatte. Somit wurde das Dashboard zuerst �ber das Ankommen des ACK informiert und dann erst �ber das Versenden.

Ein weiterer St�rfaktor ist, dass die gelisteten Zeiten von den Ger�ten selbst gemessen wurden. Da diese nicht synchronisiert wurden ensteht hier eine Ungenauigkeit. Diese ist allerdings vernachl�ssigbar, da f�r die hier durchgef�hrten Tests keine exakte Zeitbestimmung notwendig ist und durch logisches Nachdenken die korrekte Reihenfolge der Nachrichten bestimmt werden kann.
