
% Testplan --------------------------------------------------------------------

Zur Sicherstellung der Qualität der Dokumentation haben wir im Abstand von etwa zwei Wochen Reviews durchgeführt. Dabei wurde insbesondere nach Problemen in der Dokumentation gemäß der folgenden Checkliste gesucht:

\subsection{Checkliste}

\begin{itemize}
\item Gibt es Dokumentation, die nicht mehr relevant ist und entfernt werden sollte?
\item Gibt es zentrale Konzepte, zu denen keine Dokumentation existiert?
\item Ist die Dokumentation schwer verständlich (Inhalt, Rechtschreibung)?
\end{itemize}


% Durchführung der Reviews


\subsection{Review der Dokumentation vom 24. Juni 2015}

\textbf{Durchgeführt am:} 24.06.2015

\textbf{Änderungen vorgenommen am:} 26.06.2015

\paragraph{Gibt es Dokumentation, die nicht mehr relevant ist und entfernt werden sollte?}
Nein.

\paragraph{Gibt es zentrale Konzepte, zu denen keine Dokumentation existiert?}
Zur Zeit ist nur die Datenstruktur Identity dokumentiert. Dies muss ergänzt werden durch die Abstraktionsebene, welche aus dem Interface IPublicIdentity und den beiden Klassen Identity und Contact besteht, welche jeweils für die eigene Identität beziehungsweise für fremde Kontakte zuständig sind.

Es ist noch keine Dokumentation der Datenbankklasse BonfireData vorhanden, insbesondere in welchen Strukturen die Daten lokal persistent abgelegt werden.

Johannes Lauinger wird damit beauftragt, beide Probleme zu beheben. \textbf{Update 26.06.15:} Die Dokumentation wurde von Johannes Lauinger angepasst.

\paragraph{Ist die Dokumentation schwer verständlich (Inhalt, Rechtschreibung)?}
Es sollte eine Struktur, bestehend aus logisch getrennten Bereichen mit eigenen Unterpunkten, geben, sodass man in der Dokumentation schnell den gesuchten Abschnitt finden kann, um eine spezifische Frage zu beantworten.

Jonas Mönnig wird beauftragt, die Struktur zu entwickeln und einzupflegenEs sollte eine Struktur, bestehend aus logisch getrennten Bereichen mit eigenen Unterpunkten, geben, sodass man in der Dokumentation schnell den gesuchten Abschnitt finden kann, um eine spezifische Frage zu beantworten.

Jonas Mönnig wird beauftragt, die Struktur zu entwickeln und einzupflegen. \textbf{Update 26.06.15:} Dies wurde von Jonas Mönnig erledigt.


\subsection{Review der Dokumentation vom 8. Juli 2015}

\textbf{Durchgeführt am:} 08.07.2015

\textbf{Änderungen vorgenommen am:} keine Änderungen notwendig

\paragraph{Gibt es Dokumentation, die nicht mehr relevant ist und entfernt werden sollte?}
Nein.

\paragraph{Gibt es zentrale Konzepte, zu denen keine Dokumentation existiert?}
Nein.

\paragraph{Ist die Dokumentation schwer verständlich (Inhalt, Rechtschreibung)?}
Nein.


\subsection{Review der Dokumentation vom 22. Juli 2015}

\textbf{Durchgeführt am:} 22.07.2015

\textbf{Änderungen vorgenommen am:} keine Änderungen notwendig

\paragraph{Gibt es Dokumentation, die nicht mehr relevant ist und entfernt werden sollte?}
Nein.

\paragraph{Gibt es zentrale Konzepte, zu denen keine Dokumentation existiert?}
Nein.

\paragraph{Ist die Dokumentation schwer verständlich (Inhalt, Rechtschreibung)?}
Nein.


\subsection{Review der Dokumentation vom 5. August 2015}

\textbf{Durchgeführt am:} 05.08.2015

\textbf{Änderungen vorgenommen am:} keine Änderungen notwendig

\paragraph{Gibt es Dokumentation, die nicht mehr relevant ist und entfernt werden sollte?}
Nein.

\paragraph{Gibt es zentrale Konzepte, zu denen keine Dokumentation existiert?}
Nein.

\paragraph{Ist die Dokumentation schwer verständlich (Inhalt, Rechtschreibung)?}
Nein.


\subsection{Review der Dokumentation vom 18. August 2015}

\textbf{Durchgeführt am:} 18.08.2015

\textbf{Änderungen vorgenommen am:} keine Änderungen notwendig

\paragraph{Gibt es Dokumentation, die nicht mehr relevant ist und entfernt werden sollte?}
Nein.

\paragraph{Gibt es zentrale Konzepte, zu denen keine Dokumentation existiert?}
Nein.

\paragraph{Ist die Dokumentation schwer verständlich (Inhalt, Rechtschreibung)?}
Nein.


\subsection{Review der Dokumentation vom 2. September 2015}

\textbf{Durchgeführt am:} 02.09.2015

\textbf{Änderungen vorgenommen am:} keine Änderungen notwendig

\paragraph{Gibt es Dokumentation, die nicht mehr relevant ist und entfernt werden sollte?}
Nein.

\paragraph{Gibt es zentrale Konzepte, zu denen keine Dokumentation existiert?}
Nein.

\paragraph{Ist die Dokumentation schwer verständlich (Inhalt, Rechtschreibung)?}
Nein.


\subsection{Review der Dokumentation vom 14. September 2015}

\textbf{Durchgeführt am:} 14.09.2015

\textbf{Änderungen vorgenommen am:} keine Änderungen notwendig

\paragraph{Gibt es Dokumentation, die nicht mehr relevant ist und entfernt werden sollte?}
Nein.

\paragraph{Gibt es zentrale Konzepte, zu denen keine Dokumentation existiert?}
Nein.

\paragraph{Ist die Dokumentation schwer verständlich (Inhalt, Rechtschreibung)?}
Nein.


\subsection{Review der Dokumentation vom 25. September 2015}

\textbf{Durchgeführt am:} 25.09.2015

\textbf{Änderungen vorgenommen am:} keine Änderungen notwendig

\paragraph{Gibt es Dokumentation, die nicht mehr relevant ist und entfernt werden sollte?}
Nein.

\paragraph{Gibt es zentrale Konzepte, zu denen keine Dokumentation existiert?}
Nein.

\paragraph{Ist die Dokumentation schwer verständlich (Inhalt, Rechtschreibung)?}
Nein.
