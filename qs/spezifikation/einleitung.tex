
Mobile Kommunikation ist in der heutigen Zeit selbstverständlich geworden. Wir verlassen uns mehr und mehr darauf, auch von unterwegs jederzeit Nachrichten senden und empfangen zu können. Hierfür wird zum größten Teil das Handynetz verwendet.\\\\
Problematisch ist dabei, dass das Mobilfunknetz auf einen bestimmten Frequenzbereich beschränkt ist und daher nicht beliebig viele Nachrichten parallel verschickt werden können. Insbesondere bei Veranstaltungen, bei denen sich sehr viele Menschen auf relativ geringem Platz versammeln, kommt es häufig zu Netzausfällen. Ein Beispiel ist das Darmstädter Schlossgrabenfest, auf dem in den letzten Jahren ca. 100 000 Menschen anwesend waren.\\\\
Ziel unseres Projekts ist es, eine Nachrichtenplattform für solche Extremsituationen zu schaffen. Diese soll zusätzlich zur aktuell bereits häufig eingesetzten Möglichkeit, über eine Client-Server-Infrastruktur zu kommunizieren, ein Peer-to-Peer Netzwerk einsetzen. Dies bedeutet, dass die Geräte untereinander ein Netzwerk aufbauen, welches unabhängig von zentralen Netzen funktioniert, und über das Nachrichten verschickt werden können. Kann eine Nachricht nicht unmittelbar zum Empfänger geschickt werden, können weitere Geräte die Nachricht weiterleiten, bis sie am Ziel ankommt.\\\\
Die App soll sich in ihrem Funktionsumfang an verbreiteten Messenger-Anwendungen wie zum Beispiel WhatsApp orientieren. Vorerst beschränken wir uns jedoch auf einfache Textnachrichten in Chats mit nur einem Kommunikationspartner. Außerdem wird die App zunächst nur für Android entwickelt.
