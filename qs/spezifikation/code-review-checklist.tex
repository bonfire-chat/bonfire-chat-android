\documentclass[accentcolor=tud2d,12pt,paper=a4,colorbacktitle]{tudexercise}

\usepackage[utf8]{inputenc}
\usepackage{ngerman}
\usepackage{enumerate}

\title{Code Review}
\subtitle{BonfireChat}
\subsubtitle{Bachelorpraktikum 2015}

\begin{document}

  \maketitle

  \begin{enumerate}[ 1.]
    \item Ist die Funktionalität korrekt?
    \item Sind die Klassen, Funktionen und Variablen angemessen benannt?
    \item Wurde die Klassenstruktur gut entworfen und erfüllt sie alle Anforderungen, oder sind Verbesserungen nötig?
    \item Gibt es Klassen, die aufgrund neuer Implementierungen überflüssig geworden sind?
    \item Gibt es unnötig überladene Funktionen oder Konstruktoren?
    \item Wurde die Datenbankstruktur gut entworfen?
    \item Enthält das Codeteil Funktionen, die wiederverwendet werden können? Sind diese in einer Helper-Klasse untergebracht?
    \item Wurden existierende Helper-Funktionen benutzt? Ist keine doppelte Funktionalität implementiert?
    \item Werden alle Eingabeparameter validiert?
    \item Wie beeinflusst die Funktionalität die Performance der App - Stromverbrauch, Rechenzeit und Speicherbedarf?
    \item Ist es unbedingt nötig, den Code in einem UI-Thread laufen zu lassen oder würde ein Background-Thread ausreichen?
    \item Werden alle möglichen Fehlschläge behandelt?
    \item Findet eine "Graceful Degradation" statt?
    \item Werden die Best Practices zur Appentwicklung, laut Android Lint, befolgt?
    \item Welche Teile können parallel ausgeführt werden?
    \item Sind die Operationen, bei denen Threadsicherheit benötigt wird, threadsicher?
    \item Ist das Layout passend für alle Bildschirmdimensionen?
    \item Haben alle Methoden und Felder die richtigen Zugriffsmodifier?
  \end{enumerate}

\end{document}
